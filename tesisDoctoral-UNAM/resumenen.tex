\addcontentsline{toc}{chapter}{Abstract}

\begin{center}
\large{\bf Research and development of a computer vision system\\for micro machiney and micro assembly applications}

\normalsize{Doctoral Thesis}

\large{\textbf{Abstract}}

\normalsize{Gengis Kanhg Toledo Ram�rez}

{\bf Faculty of Engineering - CCADET}

{\bf National Autonomous University of Mexico}
\end{center}

The aim of this thesis is the development of a computer vision system for work pieces recognition and position location. The objective of the system is to recognize and to locate one certain work piece randomly located in a work area, so this work piece can be manipulated. The system contributes to the automatization of the machining and the assembly processes within a micro factory. Although the system had born as an automatization requeriment for a microfactory, the system can be used in any work scale.

The developed system was born from the project that consists in the creation of fully automated and shelf-reproducing microfactories. This project started at the Micromechanics and Mechatronics Laboratory of the National Autonomous University of Mexico. One fundamental requirement of this project is to achieve performance and automation for its different processes. One way to reach such requirements is to use computer vision methods. Computer vision methods allow high flexibility in the systems. For micro manipulation and micro assembly tasks, it is necessary to have a recognition and location system for work pieces. Therefore, the detailed researches in this thesis have the objetive of to develop such system allowing that these tasks will be done. 

The main goal of the researches that were done was to develop one system able to accurately recognize and to locate with acceptable precision one work piece of certain type randomly located in a work area for manipulation. In order to achieve this goal two technical aspects require to be develop, one of them is the recognition of work pieces and the other one is the location of the position of such work pieces. 

The main reach of the investigations has been to develop an off-line working system in experimental stage.

In order to manage the system develop the particular problematic was studied with special approach in the critical tasks. These tasks are the classification and the localization of work pieces. For these tasks it is fundamental to use technologies and algorithms from artificial intelligence. Two algorithms were selected, adapted and applied. These algorithms are based in artificial neural networks and code permutation paradigms. Both algorithms allowed very good performance in similar tasks. The algorithms are called neural classifier LIRA and Permutative Code Neural Classifier (PCNC).

For the analysis and proves of the system, three different databases of images from work pieces were created. The used work pieces were screws of several types, nuts, washers and other common used work pieces found in an assembly line. The images were taken from source images that contain several ramdomly located work pieces. The data bases were made with six or seven types of work pieces. The images of the databases have real characteristics actually found in the industrial enviroment such as no special illumination condition, shines, shades and not uniform background. Two databases, which are called A and B, were made with images that have only one type of work piece at a time whereas the third one, which is called D, was made with images that contains all the types of work pieces turned around and heaped up.

For the selected algorithms several experiments for validation and testing were done as well as concrete experiments with practical characteristics. 

The recognition rates reached for data bases A, B and D were, for the neuronal classifier LIRA 93.75\%, 94.14\% and 90.47\% respectively, whereas for the PCNC were 96.87\%, 97.80\% and 91.43\% respectively. The PCNC required double amount of time for image recognition with respect to the neuronal classifier LIRA. 

For both used classifiers the system obtained good performance in the recognition task. The best result of almost 98\% gives us good perspective for the developed system. The performance for the data base D, with the existence of heaped up, turned round and partially occluded work pieces is encouraging.

About the work pieces location task the system is able to provide the coordinates of the required work piece. Nevertheless to improve its precision it is necessary to use larger processing time which reduces the efficience of the proposed system. Thus, the searching and location method must be improved.

This work marks a novel research line on the computer vision problem for automation of microassembly and micromachinery. The automation of such processes means a fundamental stage in order to achieve the fully automatic microfactory. 

%%% Local Variables: 
%%% mode: latex
%%% TeX-master: "tesis"
%%% End: 
