\chapter*{\addcontentsline{toc}{chapter}{Conclusiones}Conclusiones}
En el Laboratorio de Micromec�nica y Mecatr�nica se desarrolla microequipo de bajo costo para producir dispositivos micromec�nicas. Una caracter�stica que permite el bajo costo es la utilizaci�n de algoritmos adaptivos para compensar la mayor precisi�n respecto a los equipos de alto costo. Entre los algoritmos adaptivos destacan los basados en visi�n por computadora.

...


Para la implementaci�n de los dos clasificadores neuronales, para el localizador y para la creaci�n de las bases de datos de las im�genes se desarrollaron diversos m�dulos de software. Todo este software constituy� una herramienta indispensable para la investigaci�n y los experimentos realizados. El software se desarroll� mediante un lenguaje de Programaci�n Orientado a Objetos, C++ est�ndar, desde el sistema operativo GNU/Linux.

%%% Local Variables: 
%%% mode: latex
%%% TeX-master: "tesis"
%%% End: 
