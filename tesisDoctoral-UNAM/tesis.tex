\documentclass[letter,fleqn,twoside]{book}
\usepackage{graphicx,amsmath,amsfonts,psfrag,fancyhdr,layout,appendix,subfig}
\usepackage[latin1]{inputenc}
\usepackage{makeidx}
%\usepackage{ucs}
%\usepackage[utf8]{inputenc}
%\usepackage[T2A]{fontenc} %PROBANDO para RUSO
\usepackage[sort&compress]{natbib}
%\usepackage{hypernat} CAUSA PROBLEMAS AL COMPILAR - PROBLEMAS DE MUY DIFICIL LOCALIZACI�N

%Para bibliogaf�a por cap�tulos con BibTeX
%\usepackage[sectionbib]{natbib}
%\usepackage{chapterbib}

%This change labels of subfig
\renewcommand{\thesubfigure}{\alph{subfigure}\arabic{subfiggroup}}
\captionsetup[subfigure]{labelformat=simple,labelsep=colon,
                         listofformat=subsimple}
%\captionsetup{lofdepth=2} This is in order to list the subfigures in the LOF
\makeatletter
 \renewcommand{\p@subfigure}{}
  %Esto lo agrego yo para tener subfiguras a1, b1, ... a2, b2, ... 
  %Se reinicia cada vez que una nueva figura es convocada (como es debido).
  \newcounter{subfiggroup}[figure] 
\makeatother

\usepackage[spanish]{babel}
\usepackage{epsfig}
\usepackage{url}
%Esto genera enlaces en el PDF
\usepackage{html}
%Esto es para el conversor latex2html
\usepackage{color}
\pagecolor{white}

%Este mejora las prestaciones de "\verbatim"
\usepackage{verbatim}

%Definici�n de margenes
\usepackage[left=4.5cm,top=2.8cm,right=2cm,bottom=2.5cm]{geometry}
\sloppy
\pagestyle{empty}

% Code for creating empty pages
% No headers on empty pages before new chapter
\makeatletter
\def\cleardoublepage{\clearpage\if@twoside \ifodd\c@page\else
    \hbox{}
    \thispagestyle{plain}
    \newpage
    \if@twocolumn\hbox{}\newpage\fi\fi\fi}
\makeatother \clearpage{\pagestyle{plain}\cleardoublepage}

% Code for creating fully-empty pages
% Fully empty pages before command is called
\makeatletter
\def\clearfullypage{\clearpage\if@twoside \ifodd\c@page\else
    \hbox{}
    \thispagestyle{empty}
    \newpage
    \if@twocolumn\hbox{}\newpage\fi\fi\fi}
\makeatother \clearpage{\pagestyle{empty}\clearfullypage}

% Dutch style of paragraph formatting, i.e. no indents.
\setlength{\parskip}{1.3ex plus 0.2ex minus 0.2ex}
\setlength{\parindent}{0pt}

%\includeonly{edicion}
%\includeonly{jurado,apoyo,licencia,edicion,reconocimientos,tvs}

% Double space for REVISION
%\renewcommand{\baselinestretch}{2.0}

%Print subsubsection numbers and put them in TOC
\setcounter{secnumdepth}{3}
\setcounter{tocdepth}{3}

\makeindex

%Esto no di� resultado, se deseaba pasar 0.0 -> 0.0 y no -> 0,0
%Cambia el signo decimal a punto (en lugar del predeterminado: coma)
%\DeclareMathSymbol{.}{\mathpunct}{letters}{"3B}
%\DeclareMathSymbol{,}{\mathord}{letters}{"3B}
%\DeclareMathSymbol{\decimal}{\mathord}{letters}{"3A}

\begin{document}
%Cambiar Cuadros por Tablas y lista de... Debe ir despu�s de \begin{document}
\renewcommand{\listtablename}{�ndice de tablas} 
\renewcommand{\tablename}{Tabla} 


%%%%%%%%%%%%%
\frontmatter
%%%%%%%%%%%%%
%% portada.tex-- Portada para tesis de posgrado de la Facultad de Ingenier�a. UNAM.
% Realizada por Gengis Kanhg Toledo Ram�rez (gengiskanhg.geo@yahoo.com)
%Basada en la portada escrita por: Tim Rohrer, LT, USN--31 July 1996 del NPS, Monterey, California. USA.
%y en el manual The Not So Short Introduction to LaTeX de Tobiaz Oetiker
%Eres libe de modificar este archivo de acuerdo a tus necesidades.
%13 de Junio de 2007

%Compile with:
%pdflatex portada.tex

\documentclass{book}
\usepackage[latin1]{inputenc}
\usepackage[spanish]{babel}
\usepackage{graphicx}

\setlength{\voffset}{-0.5cm}
\setlength{\hoffset}{0.7cm}
\setlength{\headsep}{0pt}
\setlength{\headheight}{0pt}
\setlength{\oddsidemargin}{-0.8in}
\setlength{\marginparwidth}{-0.5cm}
\setlength{\textwidth}{19.5cm}
\setlength{\footskip}{2pt}
\setlength{\topmargin}{0in}
\setlength{\textheight}{25cm}
\setlength{\fboxrule}{3pt}

\begin{document}
\thispagestyle{empty}

\begin{tabular}{p{3cm}p{15.0cm}}
\includegraphics[width=3cm]{figuras/escudoUNAMbn.png}
\begin{center}
\rule[2cm]{1.5mm}{13.5cm}%vertical
\hspace{2pt}
\rule[0cm]{0.7mm}{15.5cm}%vertical
\hspace{2pt}
\rule[2cm]{1.5mm}{13.5cm}%vertical
\end{center}
\includegraphics[width=2.8cm]{figuras/escudoFI-UNAM.png}
&
\vspace{-3.4cm}
\begin{center}
\Large{ \bf{UNIVERSIDAD NACIONAL AUTONOMA DE M�XICO}}
\\
\rule[0mm]{15.0cm}{0.2mm}%horizontal
\\
\rule[3mm]{15.0cm}{1.2mm}%horizontal
\\
\textbf{PROGRAMA DE MAESTR�A Y DOCTORADO EN
\\
INGENIER�A}

\vspace{2.8\baselineskip}

FACULTAD DE INGENIER�A

\vspace{2.8\baselineskip}

{\Large \bf{INVESTIGACI�N Y DESARROLLO DE SISTEMAS\\DE CONTROL MEDIANTE VISI�N T�CNICA\\PARA MICROM�QUINAS HERRAMIENTAS Y MICROMANIPULADORES}}

%\vfill
\vspace*{1.7cm}

\huge{\bf TESIS}

%\vspace*{0.1cm}
{\large QUE PARA OPTAR POR EL GRADO DE:}

%\vspace*{0.2cm}

\LARGE{\bf DOCTOR EN INGENIER�A}

%\vspace*{0.2cm}
\large{MEC�NICA - DISE�O MEC�NICO}

\vspace*{0.1cm}
P R E S E N T A:

\vspace*{1cm} {\Large \bf{GENGIS KANHG\\TOLEDO RAM�REZ$^{*}$}}

\vspace*{1.0cm}
TUTORES:

\large{\bf ERNST KUSSUL\\TATIANA BAIDYK}

\vspace*{2.0cm}
\Large{Ciudad Universitaria}\hspace*{5cm}\Large{Agosto 2007}

\end{center}

\end{tabular}

\flushright{$^{*}$Exbecario CONACYT\hspace*{0.8cm}}

%\end{center}
\newpage
\thispagestyle{empty}
$ $
\end{document} %Se compila aparte y se junta con "gs".
%gs -dNOPAUSE -sDEVICE=pdfwrite -sOUTPUTFILE=tesiscompleta.pdf -dBATCH portada.pdf tesis.pdf
\pagestyle{empty}
\newpage
\vspace{0.5cm}
\Large{\bf Jurado asignado:}

\vspace{1.3cm}
\begin{tabular}{l l}
\Large{Presidente:} & \Large{Dr. L�pez Parra Marcelo.} \\
\\
\Large{Secretario:} & \Large{Dr. Dorador Gonz�lez Jes�s Manuel.}\\
\\
\Large{$1^{er}$ Vocal:} & \Large{Dr. Kussul Ernst Mikhailovich.}\\
\\
\Large{$1^{er}$ Suplente:} & \Large{Dr. Santill�n Guti�rrez Sa�l Daniel.}\\
\\
\Large{$2^{o}$ Suplente:} & \Large{Dra. Baidyk Tatiana.}\\
\end{tabular}

\vspace{1.5cm}
Lugar donde se realiz� la tesis:

\vspace{0.5cm}
Laboratorio de Micromec�nica y Mecatr�nica (LMM) del

Centro de Ciencias Aplicadas y Desarrollo Tecnol�gico (CCADET) 

de la Universidad Nacional Autonoma de M�xico (UNAM).

\vspace{2.5cm}

\begin{center}
\Large{Tutores de tesis:}
\vspace{0.8cm}

\begin{tabular}{c p{0.5cm} c}
\rule[0mm]{5.0cm}{0.2mm} & &\rule[0mm]{5.0cm}{0.2mm} \\
\Large{\bf Dr. Ernst Kussul.} & &\Large{\bf Dra. Tatiana Baidyk.} \\ 
\end{tabular}
\end{center}
\normalsize

%%% Local Variables: 
%%% mode: latex
%%% TeX-master: "tesis"
%%% End: 

\clearfullypage
\pagenumbering{roman}
%P�gina de quienes dieron apoyo

\newpage
%\begin{flushright}
\vspace*{18cm}

\begin{tabular*}{\textwidth}{p{5cm} p{10cm}}
& Este trabajo se desarroll� en el Laboratorio de Micromec�nica y Mecatr�nica del Centro de Ciencias Aplicadas y Desarrollo Tecnol�gico (CCADET) de la Universidad Nacional Aut�noma de M�xico (UNAM) bajo la tutoria de los doctores Ernst Kussul y Tatiana Baidyk. Se cont� con el apoyo de una beca para estudios doctorales del CONACYT y de una beca complemento DGEP - UNAM. Adem�s se cont� con el apoyo parcial de los siguientes proyectos: DGAPA PAPIIT IN118799, PAPIIT IN108606-3, PAPIIT IN116306-3, NSF-CONACYT 39395-U, DGAPA-PAPIIT IN112102 y CONACYT 33944-U.
\\
\end{tabular*}

%\end{flushright}


%%% Local Variables: 
%%% mode: latex
%%% TeX-master: "tesis"
%%% End: 

%P�gina de licencia

\newpage
%\begin{flushleft}
\vspace*{19cm}
\begin{tabular*}{\textwidth}{p{10cm} p{5cm}}
El autor, sin perjuicio de la legislaci�n de la Universidad Nacional Aut�noma de M�xico, otorga el permiso para el libre uso, reproducci�n y distribuci�n de esta obra siempre que sea sin fines de lucro, se den los cr�ditos correspondientes y no sea modificada, en especial esta nota. &  \\
& \\
D.R. \copyright Gengis Kanhg Toledo Ram�rez, \hspace{1cm} M�xico, D.F. 2007. & \\
\end{tabular*}

%\end{flushleft}

%%% Local Variables: 
%%% mode: latex
%%% TeX-master: "tesis"
%%% End: 

%%% Documento realizada con LATEX, GNU, etc.
\newpage
\begin{flushright}

$ $

\vspace{18.6cm}

$ $

\rule[0mm]{4.6cm}{0.2mm}

Redacci�n y edici�n de tesis 

con \LaTeXe, \emph{GNU-Emacs} 

y sistema operativo libre 

%\emph{GNU/Linux}.

\emph{
\raisebox{-0.6ex}{G}
\raisebox{-0.1ex}{N}
\raisebox{0.3ex}{U}
\raisebox{0.0ex}{/}
\raisebox{-0.3ex}{L}
\raisebox{-0.6ex}{I}
\raisebox{-0.2ex}{N}
\raisebox{0.3ex}{U}
\raisebox{0.6ex}{X}
.}

\end{flushright}

%%%%%%% Dedicatoria
\clearfullypage
\vspace*{8cm} 

\begin{center}

\hspace{1ex} A Susy y Conenetl;

\hspace{1ex} \small{mi familia} 

\hspace{1ex} $\heartsuit$

\end{center}

%%%%%%%%% Cita
\clearfullypage
\vspace*{5cm} 

%\hspace{1ex} 
\begin{quotation} Vivimos en una sociedad altamente dependiente de la ciencia y de la tecnolog�a, en la cual casi nadie sabe nada sobre ciencia y tecnolog�a.
\flushright \emph{Carl Sagan}
\end{quotation}

\vspace*{1cm} 

\begin{quotation} Se ha vuelto espantosamente obvio que nuestra tecnolog�a ha excedido a nuestra humanidad.
\flushright \emph{Albert Einstein}
\end{quotation}

\vspace*{1cm} 
 
\begin{quotation} Las m�quinas y las computadoras deber�n volverse una parte funcional en un sistema social orientado por la vida y no un c�ncer que empieza por hacer estragos y acaba por matar al sistema.
\flushright \emph{Erich Fromm}
\end{quotation}






%%% Local Variables: 
%%% mode: latex
%%% TeX-master: "tesis"
%%% End: 

\clearfullypage
\include{reconocimientos}
%\include{russian}

% Define pagestyle
\pagestyle{fancy}
\fancyhf{}
\renewcommand{\chaptermark}[1]{\markboth{ \emph{#1}}{}}
\fancyhead[LO]{}
\fancyhead[LO]{}
\fancyfoot[LE,RO]{\thepage}

% Redefine plain page style
\fancypagestyle{plain}{
\fancyhf{}
\renewcommand{\headrulewidth}{0pt}
\fancyfoot[LE,RO]{\thepage}
}

% Dutch style of paragraph formatting, i.e. no indents.
\setlength{\parskip}{1.3ex plus 0.2ex minus 0.2ex}


% Remove parskip for toc
\setlength{\parskip}{0ex plus 0.5ex minus 0.2ex}

\tableofcontents
\listoffigures
\listoftables

\cleardoublepage
\include{resumen} 
\cleardoublepage
\addcontentsline{toc}{chapter}{Abstract}

\begin{center}
\large{\bf Research and development of a computer vision system\\for micro machiney and micro assembly applications}

\normalsize{Doctoral Thesis}

\large{\textbf{Abstract}}

\normalsize{Gengis Kanhg Toledo Ram�rez}

{\bf Faculty of Engineering - CCADET}

{\bf National Autonomous University of Mexico}
\end{center}

The aim of this thesis is the development of a computer vision system for work pieces recognition and position location. The objective of the system is to recognize and to locate one certain work piece randomly located in a work area, so this work piece can be manipulated. The system contributes to the automatization of the machining and the assembly processes within a micro factory. Although the system had born as an automatization requeriment for a microfactory, the system can be used in any work scale.

The developed system was born from the project that consists in the creation of fully automated and shelf-reproducing microfactories. This project started at the Micromechanics and Mechatronics Laboratory of the National Autonomous University of Mexico. One fundamental requirement of this project is to achieve performance and automation for its different processes. One way to reach such requirements is to use computer vision methods. Computer vision methods allow high flexibility in the systems. For micro manipulation and micro assembly tasks, it is necessary to have a recognition and location system for work pieces. Therefore, the detailed researches in this thesis have the objetive of to develop such system allowing that these tasks will be done. 

The main goal of the researches that were done was to develop one system able to accurately recognize and to locate with acceptable precision one work piece of certain type randomly located in a work area for manipulation. In order to achieve this goal two technical aspects require to be develop, one of them is the recognition of work pieces and the other one is the location of the position of such work pieces. 

The main reach of the investigations has been to develop an off-line working system in experimental stage.

In order to manage the system develop the particular problematic was studied with special approach in the critical tasks. These tasks are the classification and the localization of work pieces. For these tasks it is fundamental to use technologies and algorithms from artificial intelligence. Two algorithms were selected, adapted and applied. These algorithms are based in artificial neural networks and code permutation paradigms. Both algorithms allowed very good performance in similar tasks. The algorithms are called neural classifier LIRA and Permutative Code Neural Classifier (PCNC).

For the analysis and proves of the system, three different databases of images from work pieces were created. The used work pieces were screws of several types, nuts, washers and other common used work pieces found in an assembly line. The images were taken from source images that contain several ramdomly located work pieces. The data bases were made with six or seven types of work pieces. The images of the databases have real characteristics actually found in the industrial enviroment such as no special illumination condition, shines, shades and not uniform background. Two databases, which are called A and B, were made with images that have only one type of work piece at a time whereas the third one, which is called D, was made with images that contains all the types of work pieces turned around and heaped up.

For the selected algorithms several experiments for validation and testing were done as well as concrete experiments with practical characteristics. 

The recognition rates reached for data bases A, B and D were, for the neuronal classifier LIRA 93.75\%, 94.14\% and 90.47\% respectively, whereas for the PCNC were 96.87\%, 97.80\% and 91.43\% respectively. The PCNC required double amount of time for image recognition with respect to the neuronal classifier LIRA. 

For both used classifiers the system obtained good performance in the recognition task. The best result of almost 98\% gives us good perspective for the developed system. The performance for the data base D, with the existence of heaped up, turned round and partially occluded work pieces is encouraging.

About the work pieces location task the system is able to provide the coordinates of the required work piece. Nevertheless to improve its precision it is necessary to use larger processing time which reduces the efficience of the proposed system. Thus, the searching and location method must be improved.

This work marks a novel research line on the computer vision problem for automation of microassembly and micromachinery. The automation of such processes means a fundamental stage in order to achieve the fully automatic microfactory. 

%%% Local Variables: 
%%% mode: latex
%%% TeX-master: "tesis"
%%% End: 
 
% Adjustments headers
\fancyhead[RO]{\leftmark}

%%%%%%%%%%%%%
\mainmatter
%%%%%%%%%%%%%
\pagenumbering{arabic}
% Adjustments headers
\fancyhead[RO]{\leftmark}
\fancyhead[EL]{\emph{Cap�tulo \thechapter}}
%\setcounter{page}{3}

\include{introduction}
\include{antecedentes}
\chapter{Sistema de visi�n por computadora para microensamble}
\label{cap:SVCME}
Este cap�tulo describe el sistema de visi�n por computadora para microensamble propuesto. Dicho sistema forma parte importante de la microf�brica en desarrollo en el laboratorio de investigaci�n del autor. En el cap�tulo anterior se expuso la necesidad de que dicho sistema de visi�n sea desarrollado como parte de las investigaciones sobre microf�bricas. Se explic� por qu� un sistema de este tipo debe de estar basado en visi�n por computadora. El presente trabajo tiene como objetivo fundamental desarrollar el sistema autom�tico de visi�n por computadora de reconocimiento de piezas para una microf�brica. Para explicar lo anterior, la siguiente Secci�n (\ref{sec:LMM}) introduce el laboratorio donde se ha desarrollado este trabajo dando cuenta de sus objetivos y l�neas de investigaci�n. En la Secci�n \ref{sec:SAMP} se explica a detalle el sistema de manejo de piezas propuesto y su funci�n dentro de la microf�brica en desarrollo. Por �ltimo, en la Secci�n \ref{sec:SVT} se describe el sistema de visi�n t�cnica para reconocimiento y localizaci�n de piezas que se propone.

\section{Investigaci�n en el Laboratorio de Micromec�nica y Mecatr�nica}\label{sec:LMM}
Desde 1998, el Laboratorio de Micromec�nica y Mecatr�nica (LMM) del Centro de Ciencias Aplicadas y Desarrollo Tecnol�gico (CCADET) de la Universidad Nacional Aut�noma de M�xico (UNAM), encabezado por el Dr. Ernst Kussul, trabaja en el desarrollo de microf�bricas y las tecnolog�as relacionadas. Diversos art�culos especializados dan cuenta de la producci�n y los resultados alcanzados \citep{Rachkovskij2001,Baidyk2002,Kussul2002JMM,Kussul2002SMHS,Gengis2006,Lara-Rosano}. El LMM desde su constituci�n y hasta el momento ha tenido diversas l�neas de investigaci�n. Las principales son las siguientes:

\begin{itemize}
\item Desarrollo de microtecnolog�a mec�nica para aplicaciones en c�lulas de producci�n.
\item Investigaci�n y desarrollo de sistemas de control para microm�quinas herramientas y micromanipuladores.
\item Investigaci�n de sistemas de control basados en redes neuronales con aplicaciones en sistemas micromec�nicos.
\item Investigaci�n y desarrollo de un sistema de visi�n computacional para microequipo.
\item Investigaci�n y desarrollo de manipuladores de baja inercia.
\end{itemize}

Las bases de estas investigaciones y desarrollos fueron planteadas en \citep{Kussul1996JMM}. La propuesta general se denomina Tecnolog�a de Microequipo o MET (por sus siglas en ingl�s). Esta propuesta se ha introducido en la Secci�n \ref{sec:MET} y se ha hecho de acuerdo a lo expuesto en \citep{Kussul1993}. Mediante el uso de generaciones sucesivas de microequipo capaz de reproducirse as� mismo, pero con dimensiones m�s peque�as, se puede llegar en pocas generaciones a equipo de dimensiones microm�tricas partiendo de equipo miniatura de unos cuantos cent�metros de tama�o, v�ase Tabla \ref{t:generacionesMET}.

\begin{table*}[htb]
\caption{Generaciones de microequipo de acuerdo a su escala de reducci�n.}
\label{t:generacionesMET}\centering
\par
\begin{tabular}{|c||r|r|r|}
\hline
& \multicolumn{3}{|l|}{Escala de reducci�n (ER)} \\ \hline
Generaci�n & ER=2 & ER=4 & ER=8       \\ \hline \hline
1           & $^*$100 mm & $^*$100 mm & $^*$100 mm \\ \hline
2           & 50 mm & 25 mm & 12.5 mm \\ \hline
3           & 25 mm & 6.25 mm & 1.5 mm \\ \hline
4           & 12.5 mm & 1.5 mm & 0.2 mm \\ \hline
5           & 6 mm & 0.4 mm & 25 $\mu$m \\ \hline
6           & 3 mm & 0.1 mm & 3 $\mu$m \\ \hline
7           & 1.5 mm & 25 $\mu$m & $\triangle$ \\ \hline
8           & 0.8 mm & 6 $\mu$m & --- \\ \hline
9           & 0.4 mm & 1.5 $\mu$m & --- \\ \hline
10          & 0.2 mm & $\triangle$ & --- \\ \hline             
\multicolumn{4}{|l|}{\small{$^*$ Valor inicial igual para todos los casos.}} \\ \hline
\multicolumn{4}{|l|}{\small{$\triangle$ L�mite te�rico de micromaquinado.}} \\ \hline
\end{tabular}
\end{table*}

Adem�s del objetivo principal de crear otra generaci�n m�s peque�a para cada una de las generaciones MET, existen muchas otras aplicaciones diversas. Estas aplicaciones tienen que ver con la posibilidad de producci�n de piezas y equipos diversos de dimensiones correspondientes a cada generaci�n MET. 

Las principales �reas tecnol�gicas de aplicaci�n son la medicina, la microelectr�nica y la industria espacial. Las posibilidades son muy alentadoras ya que no s�lo se tendr� la capacidad de producir piezas y equipos de dimensiones microm�tricas, sino de cualquier dimensi�n intermedia entre lo micro y lo meso.

Sin embargo, para crear la primera generaci�n MET existen diversos problemas de ingenier�a que es necesario resolver. A medida que nuevas generaciones sean creadas nuevos problemas asociados a sus dimensiones deber�n enfrentarse debido a la miniaturizaci�n progresiva. Con el objetivo general de desarrollar el microequipo en el LMM se llevan a cabo diversas investigaciones. Estas investigaciones tienen que ver con dise�o mec�nico, precisi�n mec�nica, ingenier�a t�rmica, resistencia de materiales, interfaces, sistemas mecatr�nicos e inteligencia artificial entre otros. Todas estas investigaciones son necesarias para alcanzar los objetivos mencionados para el LMM.

Como avance concreto en el objetivo general del LMM se ha desarrollado y caracterizado un microcentro de maquinado con dimensiones de $130\times160\times85mm^{3}$. Este centro de maquinado ha permitido manufacturar objetos de entre $50\mu m$ y $5mm$. Estas piezas tienen geometr�as tridimensionales complejas, como es el caso de tornillos, engranes y ejes graduados que han sido producidos. Estas piezas producidas son de materiales diversos como acero, pl�sticos varios o lat�n \citep{Kussul2002SMHS}. Adicionalmente se han dise�ado y construido otros prototipos de la primera generaci�n MET como es el caso de un par de manipuladores. El centro de maquinado y los manipuladores se muestran en la Figura \ref{F:LMM-METmachines}.

\begin{figure}
[h]
\begin{center}
\includegraphics[
%natheight=1.030900in,
%natwidth=4.614600in,
%height=3in
width=5in
]%
{figuras/METmachinesFromLMM.jpg}%
\caption[M�quinas de la primera generaci�n MET.]{M�quinas de la primera generaci�n MET. Izquierda, microcentro de maquinado. Centro, manipulador de topolog�a com�n. Derecha, manipulador basado en paralelogramos. Un disco magn�tico de $3 \frac{1}{4}$'' se ubica en la imagen para dar cuenta de las dimensiones.}%
\label{F:LMM-METmachines}%
\end{center}
\end{figure}

Los conceptos MET y microf�brica est�n ampliamente relacionadas en la perspectiva del LMM ya que la viabilidad de esta tecnolog�a depende de la capacidad de construir una microf�brica altamente automatizada de primera generaci�n. Esta microf�brica deber� ser capaz de producir las piezas y luego las m�quinas de una siguiente generaci�n m�s peque�a como se expuso en la Sec. \ref{sec:MET}. Con el objetivo de lograr la automatizaci�n y la flexibilidad de la microf�brica se ha argumentado que una funci�n muy importante es la de un sistema autom�tico de manejo de piezas (Sec. \ref{sec:VCuF}). En la siguiente secci�n se expone la propuesta de dicho sistema enmarcando sobre �ste los objetivos del presente trabajo.

\section{Sistema autom�tico de manejo de piezas}
\label{sec:SAMP}
En una f�brica convencional los materiales y las piezas producidas pertenecen a un proceso masivo. Por lo general este proceso es poco o nada flexible durante el tiempo de producci�n. Esta forma de trabajo genera gran cantidad de productos de un mismo tipo siendo la forma dominante en el mundo desarrollado al d�a de hoy. El concepto de microf�brica no es solamente una f�brica de tama�o reducido, sino un nuevo concepto de producci�n \citep{Okazaki2004}. En el concepto de producci�n de la microf�brica se busca hacer de la producci�n algo flexible. Lo anterior significa que los productos pueden variar en dise�os y geometr�as entre muchos otros aspectos en un mismo lote de producci�n. Lo anterior hace que el material suministrado as� como los productos intermedios y terminados del proceso no sean f�cilmente manejables mediante equipos r�gidos especialmente adaptados a un s�lo producto. Es por ello que un sistema autom�tico de manejo de materiales y piezas debe ser desarrollado para la microf�brica. Independientemente de ello, un sistema as� podr� ser utilizado adem�s en las f�bricas convencionales. 

Diversas operaciones de una microf�brica requieren un sistema autom�tico de manejo de materiales y piezas, entre las que sobresalen:
\begin{itemize}
\item El suministro y manejo de materia prima de estado s�lido.
\item La manipulaci�n de los productos intermedios y finales.
\item La manipulaci�n de piezas y partes para el microensamble.
\item El remplazo y cambio de herramientas.
\item La manipulaci�n y reemplazo de partes para labores de mantenimiento.
\end{itemize}

Dada la amplia variedad de materiales y objetos a ser manejados y a las funciones requeridas, un sistema de tales caracter�sticas deber� ser altamente flexible y deber� contar con algunas capacidades atribuibles a la inteligencia, como es el caso de la identificaci�n de texturas para diferenciar el material o la determinaci�n de la posici�n de ciertos objetos aleatoriamente ordenados con miras en su manipulaci�n.

El problema descrito se aborda en el presente trabajo proponiendo un sistema autom�tico de manejo de piezas enfocado a la tarea de reconocimiento de piezas aleatoriamente distribuidas. Un sistema capaz de resolver esta compleja tarea ser� fundamental para el desarrollo de los diversos sistemas de manipulaci�n de materiales y partes para una microf�brica.

La meta de un sistema autom�tico de manejo de piezas es que sea capaz de reconocer una determinada pieza aleatoriamente ubicada en una cierta �rea de trabajo adem�s de su posici�n y orientaci�n respectiva, de tal forma que pueda ser tomada con un manipulador para ser movida a alguna otra posici�n en el espacio seg�n los requerimientos del proceso respectivo. Un sistema que cumpla con las expectativas mencionadas debe ser flexible, es decir, debe ser capaz de manejar piezas de diversas formas y con m�ltiples dimensiones.

Como soluci�n al problema planteado en el presente trabajo se tiene un Sistema Autom�tico de Manipulaci�n de Piezas (SAMP). Este sistema se ha propuesta para formar parte en una celda de ensamble en una microf�brica. El sistema propuesto se muestra en la Figura \ref{fig:piecesmanipautomaticsystem}. El SAMP est� compuesto por dos c�maras, un manipulador, un sistema de control del manipulador (SCM), un sistema inteligente de manipulaci�n (SIM) y un sistema de visi�n t�cnica (SVT).

\begin{figure}
[h]
\begin{center}
\includegraphics[
%natheight=1.030900in,
%natwidth=4.614600in,
%height=3.6832in,
width=4.0in
]%
{figuras/SistemaAutomDeManipulABloquesV2.png}%
\caption{Sistema autom�tico de manejo de piezas.}%
\label{fig:piecesmanipautomaticsystem}%
\end{center}
\end{figure}

En congruencia con la MET, el dise�o del SAMP utiliza componentes de bajo costo y de manufactura simple. La parte mec�nica del sistema es el manipulador, el cu�l debe ser dise�ado con MET para permitir su consecuente decremento en dimensi�n, un dise�o se ha propuesto en \citep{Kussul2002JMM}. Las dos c�maras del dise�o son c�maras \emph{web} de bajo costo y baja resoluci�n. Para la segunda y siguientes generaciones MET, estas c�maras deber�n emplear lentes para reducir su �rea de enfoque sin deteriorar la calidad de imagen. El resto de los componentes, los tres subsistemas propuestos, son sistemas l�gicos implementados mediante software y hardware. Estos subsistemas no requieren ser escalados en las sucesivas generaciones MET.

Cada uno de los tres subsistemas del SAMP tiene una funci�n espec�fica. El SIM es el control central del sistema de manipulaci�n. Es el encargado de comunicarse con el exterior para recibir tareas y devolver informaci�n del estado de �stas as� como del estado general del SAMP, tambi�n intercomunica los otros dos subsistemas. El SCM es la parte l�gica del manipulador, es el responsable de convertir los requerimientos dados para el SIM en t�rminos de coordenadas espaciales a el espacio de estados de la topolog�a particular del manipulador as� como realizar la planeaci�n de trayectorias de �ste. Convirtiendo los requerimientos en se�ales el�ctricas para mover los actuadores del manipulador. El SCM tambi�n devuelve el estado del manipulador, tanto espacial (posici�n, orientaci�n y estado del efector) como l�gica (banderas de error, de �xito u otros estados). El SVT es la parte inteligente del SAMP, es el responsable de devolver una posici�n espacial ante el requerimiento de localizar cierto tipo de pieza en el �rea de trabajo. Esto es, dada la necesidad de manipular una o m�s piezas, el SIM pide al SVT buscar �stas en el �rea de trabajo y devolver la posici�n de cada una de ellas para que �stas puedan ser requeridas f�sicamente al manipulador mediante el SCM. El usuario del SAMP puede ser un humano u otro sistema, e.g. sistema de manufactura o centro de control de la microf�brica.


Existen dos tareas comunes a ser resueltas por el SAMP. Una es colocar un cierto tipo de pieza ubicada en el �rea de trabajo en una microm�quina herramienta, la otra tarea es realizar un ensamble sencillo utilizando dos piezas del �rea de trabajo. Para la primera tarea el SIM pedir� al SVT localizar la pieza solicitada, en caso de que exista alguna de estas piezas en el �rea de trabajo, el SVT deber� devolver al SIM la ubicaci�n espacial de la misma. El SIM enviar� entonces esta posici�n y orientaci�n al SCM junto con los datos de destino para la pieza. La posici�n destino puede ser conocida por el SIM o no, seg�n �sta sea de una m�quina previamente ubicada o no. La segunda tarea es m�s complicada ya que requiere que el manipulador tenga capacidad para ensamble, y de no conocerse la geometr�a exacta de las piezas a ensamblar, se requerir� de un sistema de visi�n por computadora basado en la propuesta dada en \citep{Baidyk2004}. Esta segunda tarea requerir� adem�s que la tarea particular de ensamble este previamente programada de modo que el manipulador pueda colocar la primera pieza en la posici�n y lugar correctos para posteriormente ensamblarle la segunda pieza. Una vez que se logre desarrollar un sistema capaz de realizar la segunda tarea, nada impide que �ste pueda realizar ensambles con m�s piezas.

Dadas las caracter�sticas y requerimientos funcionales de cada uno de los sistemas que componen el SMAP, en el presente trabajo se ha desarrollado el SVT, ya que es el sistema que propone m�s retos y mayor dificultad, adem�s de constituir una tarea que no est� resuelta satisfactoriamente seg�n se revis� en la Secci�n \ref{sec:EdA}.

Dados los grandes retos que presenta este sistema, para realizar las investigaciones orientadas al desarrollo del SVT se procur� aislar el problema de los dem�s subsistemas y componentes del SAMP sin perder la problem�tica fundamental que se presenta. Por esto, el trabajo se hace sobre la parte del SAMP que contiene el SVT y una sola c�mara adem�s de su �rea de trabajo. Es decir, este trabajo est� basado en la tarea de reconocimiento de piezas y sus posiciones en ambiente de microf�bricas. En la siguiente secci�n final de este cap�tulo se aborda en detalle el SVT.


\section{Subsistema de visi�n t�cnica (SVT)}
\label{sec:SVT}
En las Secciones \ref{sec:RPmF} y \ref{sec:RPPmF} se han descrito los requerimientos generales para el Sistema de Visi�n T�cnica (SVT) para una microf�brica. Se tiene que un SVT para localizaci�n de piezas debe ser congruente con los principios de flexibilidad de una microf�brica. Para lo cual debe ser capaz de trabajar con diversos tipos de piezas, adem�s adaptarse a nuevas piezas en poco tiempo por lo que debe minimizarse el tiempo de procesamiento previo requerido para el reconocimiento de nuevas piezas.

Como se explic� en la secci�n anterior, el prop�sito del SVT es reconocer objetos y la posici�n de �stos. Este sistema est� compuesto de tres elementos adem�s de la c�mara. Estos elementos son una interfase, un localizador de objetos y un controlador para la c�mara (v�ase Fig. \ref{technicalvisionsubsystem}). La Interfase es la unidad que se comunica con el exterior, el SAMP (V�ase Fig. \ref{fig:piecesmanipautomaticsystem}) adem�s de intercomunicarse con los otros dos elementos del sistema. A trav�s de la Interfase el SVT recibe ordenes y env�a informaci�n. El localizador de objetos es el elemento principal del SVT, su funci�n es identificar y localizar un objeto requerido en el campo de visi�n de la c�mara, devolviendo sus coordenadas. El controlador de la c�mara se encarga de capturar im�genes del �rea de trabajo cu�ndo as� lo requiera el localizador de objetos.

\begin{figure}
[h]
\begin{center}
\includegraphics[
%natheight=1.030900in,
%natwidth=4.614600in,
%height=3.6832in,
width=2.5in
]%
{figuras/SVT.png}%
\caption{Sistema de Visi�n T�cnica.}%
\label{technicalvisionsubsystem}%
\end{center}
\end{figure}

El objetivo es encontrar las coordenadas de las piezas con una precisi�n aceptable para su posterior manipulaci�n. Esta tarea debe ser eficiente en cuanto al uso del tiempo y el uso de recursos (equipo, memoria y tiempo de procesador, etc.), de tal manera que el sistema pueda ser utilizado en l�nea de producci�n (\emph{on-line}) adem�s de ser econ�mico. En la Fig. \ref{foundworkpiecewithcoordinates} se muestra el resultado esperado por el SVT ante la solicitud de localizar un tornillo dentro del �rea de trabajo. Se busca que el SVT localice el primer objeto del tipo requerido y devuelva su posici�n y su orientaci�n. Lo ideal es que esto se haga respecto al centro de gravedad de la pieza. La posici�n y orientaci�n deben tener precisi�n tal, de forma que permitan la manipulaci�n de la pieza mediante un robot.

\begin{figure}
[h]
\begin{center}
\includegraphics[
%natheight=1.030900in,
%natwidth=4.614600in,
%height=3.6832in,
width=1.8in
]%
{figuras/piezaEncontradaConCoordenadas.png}%
\caption[Esquema de localizaci�n de una pieza reconocida.]{Ejemplo de un tornillo reconocido. El sistema devuelve las coordenadas ($x,y,\protect\theta $) con respecto del origen de la imagen.}
\label{foundworkpiecewithcoordinates}%
\end{center}
\end{figure}

Para el desarrollo del SVT han sido hechas las siguientes consideraciones sobre las piezas a reconocer y a ubicar.
\begin{itemize}
\item Se encuentran en un �rea de trabajo igual al campo visual de la c�mara.
\item Se localizan en un mismo plano\footnote{Esta consideraci�n no es del todo cierta para una de las bases de datos utilizadas, ya que las piezas est�n amontonadas y por ende, en distintos planos; sin embargo, la distancia entre �stos es relativamente peque�a.}.
\item Est�n aleatoriamente distribuidas y orientadas.
\item Pueden estar juntas e incluso una sobre otra, este caso no es general pero fue considerado para los experimentos.
\item Tienen secci�n transversal predominantemente redonda, plana o una combinaci�n de ambas caracter�sticas (en un caso), esto facilita la tarea de reconocimiento ya que piezas as� dan por lo com�n una vista superior igual sin importar su rotaci�n respecto de su eje principal. La vista superior es importante porque es la perspectiva de la c�mara del SVT.
\end{itemize}

El SVT se prob� con diversos conjuntos de piezas reales sin ning�n tipo de tratamiento especial. Las piezas escogidas son predominantemente mec�nicas las cuales se encuentran com�nmente en una l�nea de producci�n. Se utilizaron piezas tales como tornillos de diversas clases, tuercas, arandelas y otras piezas maquinadas. En la Fig. \ref{piezasUtilizadas} se muestran algunas de las piezas utilizadas. En congruencia con un ambiente de ensamble y producci�n, estas piezas en ocasiones presentan superficies sucias, tienen colores oscuros, brillo heterog�neo y a veces tienen sombras. Todas las anteriores son caracter�sticas que complican la tarea de reconocimiento.

\begin{figure}
[h]
\begin{center}
\includegraphics[
%natheight=1.030900in,
%natwidth=4.614600in,
%height=3.6832in,
width=2.5in
]%
{figuras/piezasUtilizadas.png}%
\caption[Diversas piezas utilizadas en el desarrollo del SVT.]{Algunas de las diversas piezas que se utilizaron en las investigaciones y experimentos al desarrollar el SVT.}%
\label{piezasUtilizadas}%
\end{center}
\end{figure}


El siguiente cap�tulo se dedica a la parte medular del SVT, que es el sistema de localizaci�n de piezas. Se describir�n las bases de datos de im�genes utilizadas para la experimentaci�n y se explicar�n los m�todos utilizados para la investigaci�n y desarrollo del sistema identificador de piezas adem�s de explicar el m�todo particular de localizaci�n propuesto.

%%% Local Variables: 
%%% mode: latex
%%% TeX-master: "tesis"
%%% End: 


\include{designsvt}
\include{lira}
\chapter{Clasificador Neuronal de Permutaci�n de C�digos (PCNC)}\label{cap:PCNC}
El cap�tulo que inicia est� dedicado al segundo clasificador neuronal utilizado para la implementaci�n del SVT. Este clasificador llamado
PCNC ha sido introducido en la Secci�n \ref{sec:selDeClasificadores} teniendo diferencias
sustanciales con el clasificador LIRA pero tambi�n algunas
similitudes. A continuaci�n en la siguiente secci�n se describe
detenidamente la estructura del PCNC y los procesos que lleva a
cabo. En la secci�n posterior se aborda el proceso de entrenamiento
del mismo. En la Secci�n \ref{sec:PCNCdistor} se explican las distorsiones probadas con
las base de datos utilizadas con el PCNC mientras que la Secci�n \ref{sec:PCNCdiscu} y �ltima dedica unas l�neas a la discusi�n sobre este clasificador respecto a la tarea de la que se ocupa.

\section{Estructura}\label{sec:PCNCestructura}
El PCNC est� basado en la estructura gen�rica del paradigma \emph{Associative Projective Neural Networks} (APNN) descrita en \cite{Kussul1991ANNIE,Kussul1991NNA}. Dicho paradigma incluye al \emph{clasificador de umbral aleatorio} \cite{Kussul1994AUIC,Kussul1994CS}, al \emph{clasificador neuronal de subespacio aleatorio} \cite{Kussul2001}, al clasificador LIRA \cite{Kussul2002ICVI} y al \emph{clasificador neuronal de permutaci�n de c�digos} (PCNC\footnote{PCNC son las siglas de permutative code neural classifier.}) \cite{Kussul2003IJCNN}. El PCNC al igual que el clasificador neuronal LIRA trabaja con im�genes en escala de grises.

La estructura del PCNC consta de tres partes que trabajan en forma seriada, estas son un extractor de propiedades, un codificador y un clasificador neuronal (Fig. \ref{fig:PCNC}).

\begin{figure}[h]
\begin{center} 
\includegraphics[width=\textwidth]
{figuras/PCNC.png}
\end{center}
\caption[Estructura del PCNC.]{Diagrama a bloques de la estructura del PCNC mostrando los elementos principales de intercambio entre ellos.}\label{fig:PCNC}
\end{figure}

De forma muy general el PCNC inicia su trabajo cuando una imagen en
escala de grises es presentada a la entrada del extractor de propiedades, las propiedades extra�das por este �ltimo son presentadas al codificador que las transforma en un vector binario de gran dimensi�n, vector que por �ltimo es dado al clasificador neuronal de una capa para ser procesado por �ste, ya sea para prop�sitos de entrenamiento, de prueba o para reconocimiento de alguna clase previamente entrenada.

\subsection{Extractor de propiedades}\label{ssec:PCNCextractor}
El extractor de propiedades (Fig. \ref{fig:extractor}) inicia su trabajo con una imagen en escala de grises. �ste selecciona sobre la imagen una serie de puntos espec�ficos (Fig. \ref{fig:puntosEspecificos}). Muchas formas de selecci�n de estos puntos espec�ficos pueden ser utilizadas, lo importante es que estos puntos representen las propiedades de la imagen que intervengan en su clasificaci�n o diferenciaci�n entre otras im�genes distintas a ser reconocidas. Dos m�todos para definir los puntos espec�ficos son por umbral de brillo y por extracci�n de contornos. Ambos m�todos requieren de la selecci�n de un umbral de brillo determinado $B$. Para el primer m�todo, los puntos espec�ficos de la imagen ser�n todos aquellos p�xeles cuyo brillo $b_{ij}$ sea mayor que $B$. Para el m�todo de contorno se seleccionan todos aquellos puntos en donde el gradiente del brillo, es decir los contornos, sean mayores que el umbral $B$; con esta �ltima selecci�n por umbral se logra eliminar considerablemente el ruido de la imagen de contorno.

\begin{figure}[h]
\begin{center} 
\includegraphics[width=0.8\textwidth]
{figuras/ExtractorDePropiedades.png}
\caption[Procesos del extractor de propiedades.]{Procesos del extractor de propiedades, su interrelaci�n, as� como sus entradas y salidas.}
\label{fig:extractor}
\end{center}
\end{figure}

\begin{figure}
[ht]
\centering
%\renewcommand{\thesubfigure}{\alph{subfigure})}
\setcounter{subfiggroup}{1}
\subfloat[]{\label{fig:imagenNormalizadaCono}\includegraphics[width=0.18\textwidth]{figuras/PCNC/sp/cono_0000.png}}
\subfloat[]{\label{fig:imagenNormalizadaEje}\includegraphics[width=0.18\textwidth]{figuras/PCNC/sp/eje_de_rotor_0000.png}}
\subfloat[]{\label{fig:imagenNormalizadaTerminal}\includegraphics[width=0.18\textwidth]{figuras/PCNC/sp/terminal_de_cable_0000.png}}
\\ %break current line
\setcounter{subfigure}{0} %Reset the subfigure counter 
\addtocounter{subfiggroup}{1} %My own counter is increased
\subfloat[]{\label{fig:imagenNormalizadaConoSobel}\includegraphics[width=0.18\textwidth]{figuras/PCNC/sp/cono_0000sp.png}}
\subfloat[]{\label{fig:imagenNormalizadaEjeSobel}\includegraphics[width=0.18\textwidth]{figuras/PCNC/sp/eje_de_rotor_0000sp.png}}
\subfloat[]{\label{fig:imagenNormalizadaTerminalSobel}\includegraphics[width=0.18\textwidth]{figuras/PCNC/sp/terminal_de_cable_0000sp.png}}
\caption[Selecci�n de puntos espec�ficos.]{Selecci�n de puntos
  espec�ficos con el m�todo de umbral de bordes. Se parte de una imagen normalizada y se aplica un
  operador de extracci�n de bordes Sobel. Se seleccionan aquellos puntos resultantes que son mayores que el umbral predefinido $B$. Fila superior. Im�genes
  normalizada originales. Fila inferior. Resultado de la selecci�n de
  los puntos espec�ficos sobre las im�genes respectivas de arriba, todos los p�xeles negros en la imagen son puntos espec�ficos seleccionados.}
\label{fig:puntosEspecificos}
\end{figure}

Para el presente trabajo, de acuerdo a la naturaleza de las im�genes y
las piezas que �stas contienen descritas en el Cap�tulo \ref{cap:disenoSVT}, se ha
utilizado el m�todo de extracci�n de contornos mediante un operador Sobel \cite{sobel1968}. El m�todo de umbral de brillo aplicado en im�genes
relativamente grandes como las utilizadas en este trabajo
($100\times 100$ o $150\times 150$) resulta en grandes cantidades de
puntos espec�ficos que implican el incremento masivo de recursos
necesarios para el procesamiento de �stos y para las posteriores etapas del
PCNC. 

Para cada punto espec�fico se define un rect�ngulo de dimensi�n $w\times h$ en cuyo centro est� precisamente este punto (Fig. \ref{fig:propiedades}). Desde dentro de �ste rect�ngulo se extraen m�ltiples propiedades de la imagen mediante el procedimiento explicado a continuaci�n. Un conjunto de puntos positivos $p$ y negativos $n$ determinan cada una de las propiedades dentro del rect�ngulo. Estos puntos se distribuyen aleatoriamente dentro del rect�ngulo y su n�mero es fijo para toda la estructura. Cada punto $P_{rs}$ tiene asociado un umbral $T_{rs}$ el cu�l se define aleatoriamente en el rango:
\begin{equation}
T_{min}\leq T_{rs}\leq T_{max}
\end{equation}
Los puntos positivos ser�n activos siempre que su brillo sea:
\begin{equation}
b_{rs}\geq T_{rs}.
\end{equation}
Los puntos negativos ser�n activos siempre que su brillo sea:
\begin{equation}
b_{rs}\leq T_{rs}.
\end{equation}
Una determinada propiedad existir� en el rect�ngulo si todos los
puntos tanto positivos como negativos se encuentran activos.

\begin{figure}[h]
\begin{center} 
\includegraphics[width=0.75\textwidth]
{figuras/unaPropiedadSobreUnPuntoEspecifico.png}
\end{center}
\caption[Extracci�n de propiedades.]{Extracci�n de propiedades en la imagen. Primero se definen $S$ propiedades. Cada una  mediante $p$ puntos positivos y $n$ puntos negativos aleatoriamente distribuidos sobre un rect�ngulo de $w\times h$ p�xeles. A la izquierda se ilustran ejemplos de estas propiedades con $p=4$ y $n=5$. En el cuadro principal se muestran los puntos espec�ficos de una imagen. Para cada uno de estos puntos se prueba la existencia de las $S$ propiedades. En la imagen se muestra el proceso de b�squeda de la propiedad $F_{k}$.}\label{fig:propiedades}
\end{figure}

Se procura que ninguno de estos rect�ngulos de b�squeda de
propiedades en la imagen salga de la misma. Debe recordarse que una
caracter�stica de las im�genes que se busca reconocer en este trabajo (im�genes
normalizadas) no tienen puntos de inter�s en las orillas. Sin embargo
si se elijen relativamente grandes los par�metros de ventana $w$ y $h$
con respecto a las dimensiones de la imagen $W$ y $H$ se tendr� m�s
probabilidad de que existan rect�ngulos que salgan de los l�mites de
la imagen, esto se hace m�s probable si existen puntos espec�ficos
cerca de las orillas. Considerando lo anterior se expande la imagen $w/2$ p�xeles a cada lado y $h/2$ p�xeles arriba y abajo con color
blanco, es decir, significando ausencia de todo punto espec�fico
posible y evitando que cualquiera de estos rect�ngulos salga de la imagen ampliada. 

Se utilizan muchas propiedades distintas $F_{i} \; | \; i\in [1,S]$. Donde
$S$ es por lo general del orden de unidades de millar. El extractor de propiedades
examina las $S$ propiedades para cada uno de los puntos espec�ficos
definidos. Todas las propiedades extra�das de todos los puntos espec�ficos definidos son entregados al codificador.

\subsection{Codificador}\label{ssec:PCNCcodificador}
El codificador (Fig. \ref{fig:codificador}) transforma las propiedades dadas por el extractor de propiedades a un vector
binario:
\begin{equation}
V=\left\{v_{i} \; | \; v_{i}=\{0,1\}, \; i\in(1,N)\right\},
\end{equation}

\begin{figure}[h]
\begin{center} 
\includegraphics[width=\textwidth]
{figuras/codificador.png}
\end{center}
\caption[Procesos del codificador.]{Principales procesos que lleva a cabo el codificador as� como sus entradas y salidas.}\label{fig:codificador}
\end{figure}


Para cada propiedad extra�da $F_{k}$ el
codificador crea un vector adicional binario:
\begin{equation}
U_{k}=\left\{u_{i} \; | \; u_{i}=\{0,1\}, \: i\in(1,N)\right\},
\end{equation}
Este vector contiene K 1's, donde $K\ll N$, al menos mil veces menor. Un procedimiento aleatorio que se explica m�s tarde es utilizado para elegir las posiciones de los unos en el vector U para cada propiedad $F_{k}$. Este procedimiento genera la lista de posiciones de unos para cada caracter�stica y salva todas estas listas en memoria no vol�til. El vector $U_{k}$ es llamado \emph{m�scara} de la propiedad $F_{k}$.

En la siguiente etapa del proceso de codificaci�n se hace necesario
transformar el vector auxiliar $U$ al nuevo vector $U^{*}$ el cual
corresponde a la propiedad localizada en la imagen. Esta
transformaci�n se hace mediante permutaciones de los componente del
vector $U$. El n�mero de permutaciones depende de la localizaci�n de la propiedad en la imagen. Las permutaciones correspondientes a las direcciones horizontal ($X$) y vertical ($Y$) son permutaciones diferentes.

Una permutaci�n de $m$ elementos $P^{m}$ puede ser representada como un vector $m$-dimensional. Para aplicar esta permutaci�n a un vector �ste debe ser de dimensi�n $m$. En t�rminos formales:

\begin{equation}
P^{m}(V)=V' \; | \; V'=\{v'_{i}\} \; , \; v'_{p_{i}}=v_{i} \; , \quad P,V,V'\in \Re^{m}
\end{equation}

Lo cual significa que el resultado de aplicar la permutaci�n $P^{m}$ a un vector $V$ resulta en un nuevo vector $V'$ cuyos componentes se definen tomando cada componente $v_{i}$ de $V$ y coloc�ndolo en la posici�n $p_{i}$ de $V'$, es decir, a la que apunta el �ndice correspondiente del vector de permutaci�n $P^{m}$. En la Fig. \ref{fig:permutacion} se ilustra un ejemplo gr�fico simple a este respecto.

\begin{figure}[h]
\begin{center} 
\includegraphics[width=0.5\textwidth]
{figuras/permutacionYsuRepresentacion.png}
\end{center}
\caption[Permutaci�n y su representaci�n.]{Permutaci�n y su representaci�n sobre un vector. Se ilustra una permutaci�n $P^{6}$ como un vector. El resultado de aplicar esta permutaci�n sobre un vector $V$ se ilustra a la derecha. Cada elemento $v_{i}$ de $V$ es reordenado en la posici�n se�alada por la componente correspondiente de $P^{6}$, dando como resultado un nuevo vector $V'$ con los mismos elementos de $V$ pero en orden distinto.}\label{fig:permutacion}
\end{figure}

\subsubsection{Codificaci�n de las propiedades}
El problema a resolver es obtener c�digos binarios de las propiedades extra�das los
cuales tengan correlaci�n fuerte si la distancia entre las
localizaciones de las propiedades es peque�a y tengan correlaci�n
d�bil o no tengan ninguna si esta distancia es grande. Por ejemplo, si una misma propiedad $F_{k}$ se extrae tanto en un extremo de la pieza en la imagen como en el otro extremo entonces �stas deben ser codificadas como vectores binarios distintos $U^{*}_{k1}$ y $U^{*}_{k2}$, existiendo entre estos correlaci�n d�bil o ausencia alguna de correlaci�n. En el caso de que la misma propiedad sea encontrada en puntos vecinos entonces estas deben ser codificadas con los mismos vectores $U^{*}_{k3}$ y $U^{*}_{k4}$. Esta propiedad que se acaba de describir permite que el sistema de reconocimiento sea insensible a peque�os desplazamientos de los objetos en la imagen.

Para codificar la localizaci�n de la propiedad $F_{k}$ en la imagen es necesario definir y dar valor a la distancia de correlaci�n $D_{c}$. Sea la propiedad $F_{k}$ detectada en dos puntos distintos $P_{1}(x_{1},y_{1})$ y $P_{2}(x_{2},y_{2})$, sean $U^{*}_{P1}$ y $U^{*}_{P2}$ los vectores binarios que codifican a $F_{k}$ para estos puntos respectivamente y sea $d$ la distancia euclidiana entre estos puntos dada por:
\begin{equation}
d=\sqrt{(x_{2}-x_{1})^{2}+(y_{2}-y_{1})^{2}}
\end{equation}
Se requiere que:
\begin{equation}
d < D_{c} \Rightarrow U^{*}_{P1} \;\mbox{y}\; U^{*}_{P2} \quad\mbox{est�n correlacionados y,} 
\end{equation}
\begin{equation}
d \geq D_{c} \Rightarrow U^{*}_{P1} \;\mbox{y}\; U^{*}_{P2} \quad\mbox{no est�n correlacionados.}
\end{equation}

Para buscar cumplir con estas propiedades se calculan los siguientes valores para una propiedad detectada $F_{k}$ en un punto $P(i,j)$:

\begin{eqnarray}
X=i/D_{c},\nonumber\\
E(X)=int\left( X \right) ,\label{eq:permX}\\
R(X)=i-E(X)\cdot D_{c},\nonumber
\end{eqnarray}

\begin{eqnarray}
Y=j/D_{c},\nonumber\\
E(Y)=int\left( Y \right),\label{eq:permY}\\
R(Y)=j-E(Y)\cdot D_{c},\nonumber
\end{eqnarray}

\begin{equation}
p_{x}=int\left(\frac{R(X)}{D_{c}}N \right),
\label{eq:permFracX}
\end{equation}
\begin{equation}
p_{y}=int\left( \frac{R(Y)}{D_{c}}N \right),
\label{eq:permFracY}
\end{equation}

donde $int()$ es la funci�n entero. Por lo tanto $E(X)$ y $E(Y)$ son las partes enteras de $X$ y $Y$ respectivamente y $R(X)$ y $R(Y)$ son las partes fraccionarias de $X$ y $Y$ respectivamente.

La m�scara (vector $U_{k}$) de la propiedad $F_{k}$ se considera como un c�digo de esta propiedad localizado en el punto origen de la imagen: $O(0,0)$. Se definen las permutaciones $\mathbf{P_{x}}$ y $\mathbf{P_{y}}$ como requisito para obtener los c�digos correspondientes a las propiedades existentes en cualquier punto de la imagen fuera del origen. En general, para obtener el c�digo de la propiedad $F_{k}$ perteneciente al punto $P(i,j)$ se procede de la siguiente forma:

\begin{enumerate}
\item Primero se trata el desplazamiento horizontal, para lo cual se aplica $E(X)$ veces la permutaci�n $\mathbf{P_{x}}$ al vector $U_{k}$, despu�s se aplica la misma permutaci�n una vez m�s pero solamente a los primeros $p_{x}$ componentes del vector $U_{k}$.
\item Segundo, se trata el desplazamiento vertical de forma an�loga. Se aplica la permutaci�n $\mathbf{P_{y}}$ $E(Y)$ veces con una permutaci�n adicional para los primeros $p_{y}$ componentes de $U_{k}$.
\end{enumerate}

\subsubsection{Ejemplo de permutaciones}
El procedimiento anterior se ilustra con un ejemplo para facilitar su comprensi�n. La dimensi�n del vector $U$ y los valores $x$ y $y$ del punto $P$ se eligen peque�os con el prop�sito de dar claridad al proceso. Tomemos como par�metros del PCNC $D_{c}=6$ y $N=8$ y sean $\mathbf{P_{x}}$ y $\mathbf{P_{y}}$ dos permutaciones de dimensi�n $N$. 

Sup�ngase detectada la propiedad $F$ en el punto $P(10,14)$ y sea el vector $U$ la m�scara correspondiente a esta propiedad. La tarea consiste en codificar $U$ en un nuevo vector $U^{*}$ atendiendo a las coordenadas del punto $P$. Como primer paso se aplican \eqref{eq:permX}, \eqref{eq:permY}, \eqref{eq:permFracX} y \eqref{eq:permFracY} resultando: $E(X)=1$, $E(Y)=2$, $p_{x}=5$ y $p_{y}=2$. Se aplica a continuaci�n $E(X)$=1 permutaci�n $\mathbf{P_{x}}$ al vector $U$ (Fig. \ref{fig:permutacionesX}) y luego al resultado ($U_{1}$) se aplica una permutaci�n adicional �nicamente a los primeros $p_{x}=5$ componentes obteniendo con esto el vector $U'$, todas las componentes que no sean definidas mediante la permutaci�n parcial se copian del vector anterior correspondiente no importante si fueron permutados o no. Con el proceso anterior tenemos las siguientes trayectorias de ejemplo: $u_{1} \rightarrow u_{1,3} \rightarrow u'_{4}$, $u_{2} \rightarrow u_{1,7} \rightarrow$ se elimina y $u_{8} \rightarrow u_{1,5} \rightarrow u'_{8},u'_{5}$. Tenemos que para el primer ejemplo el valor de $u_{1}$ es permutado dos veces y termina en $u'_{4}$. Para el segundo caso $u_{2}$ se permuta una sola vez a $u_{1,7}$ y luego se ignora porque a la componente $7$ es menor que $p_{x}=5$ y por que la componente correspondiente en $U'$ ya ha sido ocupada por $u_{7}$. Para el tercer caso $u_{8}$ se permuta las dos veces pero adicionalmente es copiado a la componente $u'_{5}$ pues luego de permutar los primeros $5$ elementos de $U'$ queda vac�a. 
\begin{figure}[h]
\begin{center} 
\includegraphics[width=0.75\textwidth]
{figuras/permutacionesXejemplo.png}
\end{center}
\caption[Permutaciones X.]{Ejemplo de permutaciones $\mathbf{P_{X}}$ sobre el vector $U$ con $E(X)=1$ y $p_{x}=5$. Se aplica una vez la permutaci�n a todo el vector y una vez m�s s�lo a los primeros cinco componentes del mismo. En este sentido las l�neas punteadas representan cambios que no deben realizarse.}\label{fig:permutacionesX}
\end{figure}

En este ejemplo se han dado los tres casos posibles que pueden ocurrir, sin embargo debe tenerse en cuenta que de acuerdo a la naturaleza pr�ctica de los vectores m�scaras ($U$) que tienen dimensi�n $N$ muy grande y sus componentes son mayoritariamente ceros, los casos especiales de eliminaci�n ocurren muy rara vez como se explicar�.

Una vez habiendo realizado las permutaciones $X$, realizamos las permutaciones $Y$ con el mismo procedimiento, s�lo que ahora lo haremos con el vector $U'$ y aplicaremos la permutaci�n $\mathbf{P_{y}}$ y los par�metros $E(Y)$ y $p_{y}$. En la Fig. \ref{fig:permutacionesY} se muestra gr�ficamente este procedimiento. El resultado de esta permutaci�n es el resultado de ambas permutaciones aplicadas sobre el vector m�scara $U$ de la propiedad $F$ y le llamamos vector $U^{*}$. Este vector codifica la propiedad $F$ en la ubicaci�n del punto $P(10,14)$ de la imagen correspondiente.

\begin{figure}[h]
\begin{center} 
\includegraphics[width=0.75\textwidth]
{figuras/permutacionesYejemplo.png}
\end{center}
\caption[Permutaciones Y.]{Ejemplo de permutaciones $\mathbf{P_{Y}}$ aplicado sobre el vector resultante de las permutaciones $\mathbf{P_{X}}$ con $E(Y)=2$ y $p_{y}=2$. La permutaci�n se aplica dos veces a todo el vector y una vez m�s solamente a los primeros dos elementos del mismo. Las l�neas punteadas de la �ltima permutaci�n indican cambios que no deben realizarse.}\label{fig:permutacionesY}
\end{figure}


\subsubsection{Propiedades de las permutaciones}
Consideremos ahora las propiedades de las permutaciones descritas. Sup�ngase que la propiedad $F_{k}$ ha sido detectada en dos puntos $P_{1}(x_{1},y_{1})$ y $P_{2}(x_{2},y_{2})$tales que $x_{1}\neq x_{2}$ y $y_{1}\neq y_{2}$. Sea $U_{k}$ la m�scara correspondiente para esta propiedad. Se definen $d_{x}$ y $d_{y}$ como:
\begin{equation}
d_{x}=|x_{2}-x_{1}|,
\end{equation} 
\begin{equation}
d_{y}=|y_{2}-y_{1}|
\end{equation} 

Suponiendo que $dx\neq 0$, luego de realizar las permutaciones horizontales ($\mathbf{P_{x}}$) los vectores correspondientes $U_{1}$ y $U_{2}$ ser�n distintos. Sea $\Delta n$ la diferencia en el n�mero de 1's entre los vectores. Puede mostrarse que el valor promedio de $\Delta n$ puede calcularse de forma aproximada con:

\begin{equation}
\Delta n \approx \frac{K}{D_{c}}d_{x} 
\end{equation}

donde $K$ es el n�mero de unos del vector auxiliar binario $U$ de la propiedad $F_{k}$. Luego de las permutaciones verticales ($\mathbf{P_{y}}$) los vectores resultantes correspondientes $U^{*}_{1}$ y $U^{*}_{2}$ tendr�n diferencias que pueden ser estimadas por:

\begin{equation}
\overline{\Delta n} \approx K\left(1-\left(1-\frac{d_{x}}{D_{c}}\right)\left(1-\frac{d_{y}}{D_{c}}\right)\right), \; 
1>\overline{\Delta n}>0.
\end{equation}

Por lo anterior los vectores $U^{*}_{1}$ y $U^{*}_{2}$ estar�n correlacionados solamente si se cumple $d_{x}<D_{c}$ y $d_{y}<D_{c}$. La correlaci�n se incrementar� si $d_{x}$ y $d_{y}$ se decrementan.

Puede verse en la Fig. \ref{fig:permutacionesXY} que distintos componentes del vector $U$ pueden pretender quedar en la misma posici�n luego de realizar las permutaciones. Por ejemplo, luego de las permutaciones $\mathbf{P_{x}}$ el componente $u_{8}$ ocupa dos posiciones en $U'$ o luego de aplicar ambas permutaciones el componente $u_{6}$ termina en dos posiciones y el $u_{8}$ en tres. Estos eventos son indeseables y no son un problema para el PCNC pues la probabilidad de que tales eventos indeseables sucedan est� relacionada inversamente a la dimensi�n de $N$ y a la relaci�n $N/K$ y atendiendo a los valores grandes de $N$ y peque�os para $K$ la probabilidad de estos eventos es menor de 0.01\% \cite{Kussul2006IEEE}.

\begin{figure}[h]
\begin{center} 
\includegraphics[width=0.75\textwidth]
{figuras/permutacionesXYejemplo.png}
\end{center}
\caption[Permutaciones XY.]{Resultados obtenidos al aplicar las permutaciones $\mathbf{P_{x}}$ sobre el vector $U$ y luego la permutaci�n $\mathbf{P_{y}}$ a ese resultado $U'$. En la figura se muestra el orden final de las componentes del vector $U$ dentro del vector final $U^{*}$.}\label{fig:permutacionesXY}
\end{figure}

\subsubsection{Vector c�digo $\mathbf{V}$ }
Una vez calculados todos los vectores $U^{*}_{r}$ de todas las propiedades detectadas en la imagen se crea el vector c�digo definido como:

\begin{equation}
V=\left\{v_{i}\ \; | \; v_{i}= \bigwedge u^{*}_{ri} \; , \; i\in(1,N)\right\}
\label{eq:vectorCodigo}
\end{equation}

donde $\bigwedge$ es el s�mbolo de la disyunci�n, $u^{*}_{ri}$ es el $i$-�simo componente del vector $U^{*}_{r}$, vector que es el corresponde a la propiedad detectada $F_{r}$.

Al utilizar n�meros aleatorios independientes para la generaci�n de las m�scaras se logra que este proceso de codificaci�n produzca representaciones mayoritariamente independientes para todas las propiedades. La �nica pero d�bil influencia entre las distintas propiedades aparece cuando se absorbe alg�n \emph{1} en la disyunci�n, Eq. \eqref{eq:vectorCodigo}.

\subsubsection{Adelgazador dependiente de contexto (CDT)}\label{sssec:CDT}
Para poder reconocer alguna pieza en una imagen se hace necesario
utilizar combinaci�n de propiedades, es decir, combinar la existencia
de ciertas propiedades con la ausencia de otras. Para lograr este
prop�sito de combinaci�n de propiedades se ha utilizado con �xito el
CDT\footnote{Llamado as� por ser las siglas en ingl�s de Context
  Dependent Thinning.} o \emph{Adelgazador dependiente de contexto.}
\cite{Rachkovskij2001}. El CDT ha sido desarrollado en base al proceso
de normalizaci�n de vectores \cite{Artikutsa1991}. Si bien existen
diversos procedimientos de implementaci�n del CDT en este trabajo se
utiliza el procedimiento ilustrado en la
Fig. \ref{fig:permutacionesQ}, el cu�l requiere de un n�mero entero
$q$ como par�metro de entrada y consiste en lo siguiente:

\begin{enumerate}
\item Se genera una nueva permutaci�n $\mathbf{Q}$ de dimensi�n $N$ la cu�l es independiente de $\mathbf{P_{x}}$ y $\mathbf{P_{y}}$.
\item Se prueba cada componente $v_{i}$ del vector $V$,
  si $v_{i}=0$ no se hace nada
  si $v_{i}=1$ se considera la trayectoria de este componente individual durante $q$ permutaciones $\mathbf{Q}$. Si esta trayectoria pasa por al menos un elemento \emph{1} del vector $V$, el valor de $v_{i}$ se hace 0.
\item Al vector resultante se le llama $V'$.
\end{enumerate}

Un ejemplo gr�fico de lo anterior se tiene en la Fig. \ref{fig:permutacionesQ} donde $q=4$. La permutaci�n $\mathbf Q$ se representa por las flechas. Para el elemento $v_{1}$ se sigue la trayectoria indicada por la permutaci�n la cu�l es: $v_{1} \rightarrow v_{7} \rightarrow v_{8} \rightarrow v_{5}$, si cualquiera de las componentes $v_{7}$, $v_{8}$ o $v_{5}$ es igual a $1$ entonces $v_{1}=0$. Lo mismo se hace para todos los elementos de $V$, construy�ndose un nuevo vector sin alterar el vector original.

\begin{figure}[h]
\begin{center} 
\includegraphics[width=0.75\textwidth]
{figuras/permutacionesQejemplo.png}
\end{center}
\caption[Permutaciones Q.]{Aplicaci�n de $q$ permutaciones $\mathbf{Q}$ sobre el vector $V$.}\label{fig:permutacionesQ}
\end{figure}

El n�mero $q$ es un par�metro de reconocimiento del sistema. Una vez que se aplica el CDT al vector $V$ se ha cumplido con el proceso correspondiente del codificador. El vector $V'$ de dimensi�n $N$ representa el c�digo de la imagen presentada originalmente al extractor de propiedades. Este resultado est� listo para ser pasado al clasificador neuronal.

\subsection{Clasificador neuronal}\label{ssec:PCNCnc}
En la Secci�n \ref{sec:selDeClasificadores} se ha mencionado y hecho referencia al perceptr�n de una capa de Rosenblatt. Este perceptr�n posee muy buena convergencia sin embargo requiere que en el espacio param�trico las clases tengan separabilidad lineal. Para obtener esta separabilidad lineal, las etapas anteriores del PCNC, el extractor de propiedades y el codificador, convierten el espacio param�trico de una imagen que es representado por el brillo de todos los p�xeles de �sta, a un espacio param�trico de mayor dimensi�n. En general se tiene un espacio param�trico de dimensi�n $W\cdot H$ convertido a otro de dimensi�n $N$, donde $W$ y $H$ son el ancho y el alto en p�xeles de las im�genes a procesar por el PCNC y $N$ es el par�metro del PCNC igual a la dimensi�n del vector $V'$. Este procedimiento descrito mejora considerablemente la separabilidad lineal del espacio param�trico que representa la imagen y el objeto que esta contiene.

En la Fig. \ref{fig:liraDividido} se presenta de nuevo la estructura del clasificador neuronal LIRA pero dividi�ndolo. Obs�rvese que las primeras tres capas $S$, $I$ y $A$ cumplen la funci�n de un extractor de propiedades al ser las entradas respectivas de cada grupo de la capa $I$ seleccionadas aleatoriamente dentro de un rect�ngulo posicionado aleatoriamente en la imagen (Sec. \ref{sec:LIRAestructura}). En cambio, las salidas de la capa $A$ y la capa $R$ en su totalidad, cumplen la funci�n propia del clasificador neuronal cuya estructura est� basada como ya se explic� en el perceptr�n. De lo que se trata es de utilizar esta segunda parte de la estructura junto con el extractor de propiedades y el codificador (Fig. \ref{fig:PCNC}). Es decir, respecto del clasificador LIRA, se sustituyen las capas $S$, $I$ y $A$ por los dos bloques previamente descritos del PCNC. Debe notarse que con esta sustituci�n, la capa $A$ del clasificador neuronal pasa a contener exactamente al vector $V'$ por lo cu�l debe tener $N$ elementos.

\begin{figure}[h]
\begin{center} 
\includegraphics[width=0.75\textwidth]
{figuras/liragsEspDividida.png}
\end{center}
\caption[Clasificador neuronal LIRA dividido.]{Clasificador neuronal LIRA dividido en dos partes seg�n su funci�n part�culas: extractor de propiedades y clasificador.}\label{fig:liraDividido}
\end{figure}

Una vez que se ha descrito con todo detalle la estructura del PCNC se explicar� su proceso de entrenamiento.

\section{Proceso de entrenamiento}
De acuerdo a las similitudes entre las estructuras del PCNC y del clasificador neuronal LIRA se tiene que la �nica parte de ambos que varia durante el proceso de entrenamiento son las conexiones entre las capas $A$ y R, por lo tanto el proceso de entrenamiento descrito para el clasificador LIRA (Sec. \ref{sec:LIRAentrenamiento}) funciona igual para el PCNC y por lo tanto es aplicado. El proceso de codificaci�n de las im�genes aplicado a LIRA es v�lido para el PCNC por lo que se aplica tambi�n. Esto se debe a que cada imagen tendr� un vector $V'$ que codifica sus propiedades extra�das por lo cu�l pueden usarse �stas a trav�s de $V'$ en lugar de la imagen con prop�sitos de entrenamiento ahorrando importantes recursos de computo y de tiempo en el proceso.

\section{Distorsiones}\label{sec:PCNCdistor}
De forma similar que con el clasificador neuronal LIRA y con el objetivo de ampliar el conjunto de im�genes destinadas para el entrenamiento del PCNC, se consider� la aplicaci�n de distorsiones sobre las im�genes normalizadas originales. Atendiendo a que el PCNC no es sensible a desplazamientos cartesianos de la imagen, s�lo se aplicaron distorsiones angulares (Fig. \ref{fig:distorsionesTeta}). Estas distorsiones aplicadas sobre las im�genes, representan peque�as variaciones sobre la exacta alineaci�n de las piezas con respecto al eje horizontal que pasa por el centro de la imagen. Por esto se realizaron experimentos con diversas distorsiones angulares. Estos experimentos se explican en el siguiente cap�tulo, en la Secci�n \ref{ssec:PCNCdistor}.

Para realizar estas distorsiones se ha utilizado el mismo m�todo de creaci�n de im�genes distorsionadas que para el clasificador LIRA. Se consider� la creaci�n de tales distorsiones en pares, con un mismo valor absoluto angular, pero con signos opuestos.

\begin{figure}[!h]
\begin{center} \includegraphics[
width=1.8in
]{figuras/distorsionesTeta.png}\end{center}
\caption[Distorsiones para ampliar el conjunto de entrenamiento.]{Ejemplo de una distorsi�n angular para las im�genes del conjunto de entrenamiento destinadas al PCNC.}
\label{fig:distorsionesTeta}
\end{figure}

\section{Discusi�n}\label{sec:PCNCdiscu}
Se ha visto que el clasificador PCNC es igual que el LIRA en cuando a
su m�todo de clasificaci�n, variando en ambos la forma en que se crean
o extraen las propiedades de la imagen a clasificar. Es entonces en
los procesos de extracci�n de propiedades y de codificaci�n donde est�
la diferencia entre el PCNC y LIRA. Esta diferencia tiene dos
caracter�sticas de importancia en el PCNC, una de �stas es la capacidad de
aplicar y probar diversos m�todos de extracci�n de puntos espec�ficos
y no limitarse �nicamente a niveles de brillo a trav�s de un umbral,
abriendo paso de esta manera a las posibilidades que ofrece el preprocesamiento de las im�genes a clasificar; la otra caracter�stica medular es la
consideraci�n expl�cita que lleva a cabo el codificador sobre la posici�n de las propiedades encontradas en la imagen.

La principal desventaja del PCNC sobre el clasificador LIRA est� en
sus mayores requerimientos de c�mputo debido a la necesidad de
realizar gran cantidad de c�lculos con vectores durante el proceso de
codificaci�n de cada imagen. Comparando la etapa de extracci�n de propiedades de LIRA y del PCNC se observa que ambos ubican ventanas aleatoriamente en la imagen de entrada, el n�mero de puntos de cada una de ellas puede ser similar, sin embargo, mientras la cantidad total de ventanas aleatorias en LIRA ser� igual al n�mero de grupos en la capa $I$, que es un par�metro de este clasificador del orden de $100 000$ para la tarea de reconocimiento que nos ocupa \cite{Gengis2004}, se tiene que en el PCNC el n�mero de tales ventanas ser� el n�mero de puntos espec�ficos extra�dos de la imagen multiplicado por el par�metro $S$, y dado que este par�metro est� entre $1000$ y $10 000$ \cite{Kussul2006IEEE}, tenemos que para el caso de $S=1000$ con $100$ puntos espec�ficos igualamos los requerimientos del clasificador LIRA y en el orden en que este �ltimo n�mero sea rebasado lo ser�n los requerimientos del PCNC respecto de LIRA en t�rminos generales. Si consideramos que las im�genes normalizadas con que se ha trabajado (Sec. \ref{ssec:ImN}) tienen dimensi�n de $100\times 100$ como m�nimo, igual a $10 000$ p�xeles se tiene que el n�mero de puntos espec�ficos para igualar los requerimientos de LIRA deber�n ser de aproximadamente el 1\%. Esto lleva a intentar mejorar el m�todo de selecci�n de puntos espec�ficos o a tener que trabajar con las im�genes normalizadas reducidas cierta escala. Por lo anterior, de usarse el tama�o original de las im�genes, el m�todo de selecci�n de puntos espec�ficos por umbral arrojar� gran porcentaje de puntos respecto al total de p�xeles de la imagen por lo que los requerimientos computacionales y de tiempo crecer�n tanto que se violar�n los principios de econom�a establecidos para las microf�bricas y el SVT (Sec. \ref{sec:mF} y \ref{sec:SVT}). El inconveniente anterior se resolvi� utilizando el umbral de gradiente de brillo (extracci�n de contornos), lo cual ha reducido el n�mero de puntos espec�ficos significativamente sin que se haya traducido en reducci�n de las propiedades que posibilitan la adecuada clasificaci�n de las im�genes. Prueba de lo anterior se presenta en el cap�tulo siguiente y �ltimo.

%%% Local Variables: 
%%% mode: latex
%%% TeX-master: "tesis"
%%% End: 



\chapter{Experimentos y resultados}\label{Cap:Exp}
%This change labels of subfig
\renewcommand{\thesubfigure}{\alph{subfigure}}
\captionsetup[subfigure]{labelformat=simple,labelsep=colon,
                         listofformat=subsimple}
%FIN de configuraci�n

Este cap�tulo final se dedica a los experimentos realizados con los
clasificadores LIRA y PCNC sobre las bases de datos descritas en la Secci�n \ref{ssec:BDI}. La
primera secci�n de este cap�tulo se dedica a los experimentos con el clasificador neuronal LIRA y la segunda a los experimentos con el PCNC. La tercer secci�n compara los resultados de ambos clasificadores y discute sobre de �stos. La cuarta y �ltima secci�n se dedica a la tarea de b�squeda y localizaci�n de piezas.

En este trabajo un experimento significa una serie de pruebas con el clasificador LIRA o PCNC encaminadas a probar o concluir una propiedad u  objetivo particular del mismo.

Para ambos clasificadores neuronales se realizaron m�ltiples
experimentos con las bases de datos descritas. Los
experimentos llevados a cabo han sido objetivos, sistem�ticos y
estad�sticamente convincentes. Los experimentos son objetivos por que
parten de un plan predeterminado y bien definido para su realizaci�n
el cu�l se explica en breve. Son sistem�ticos por que se han hecho de
forma ordenada y organizados en etapas, utilizando cuando es necesario
los resultados obtenidos en los experimentos previos en los
subsecuentes. Y por �ltimo los experimentos son estad�sticamente
convincentes por que se ha tenido cuidado de ejecutar m�ltiples
pruebas agrupadas en experimentos con metodolog�a id�ntica.

Todos los experimentos descritos en este trabajo se ejecutaron en un computador con procesador Intel$^{\copyright}$ Pentium$^{\copyright}$ 4 a 2.80 GHz con 512 KB de memoria cach� y 512 MB de memoria RAM con sistema operativo GNU/Linux kernel 2.6.17-11-generic.

\section{LIRA}
Primero que nada es importante se�alar que dada la estructura del clasificador LIRA
(Sec. \ref{sec:LIRAestructura}), el orden de magnitud en los valores pr�cticos de
algunos de sus par�metros ($W\times H$, $w\times h$ y $N$), y por el
car�cter aleatorio de su construcci�n es sumamente improbable que dos
estructuras id�nticas LIRA sean creadas, esto considerando id�nticos
par�metros de creaci�n. Por lo anterior es importante mencionar cuando
un experimento utiliz� un mismo clasificador LIRA (misma creaci�n) y
cuando se utilizaron distintos clasificadores con id�nticos
par�metros. Adicionalmente debido al proceso de entrenamiento del
clasificador LIRA (Sec. \ref{sec:LIRAentrenamiento}), se tiene que un mismo clasificador
puede estar entrenado con diversos conjuntos de entrenamiento y
diversos ciclos de entrenamiento, por lo cu�l a�n copias id�nticas de
un mismo clasificador pueden presentar respuestas distintas. Tener
presente lo anterior ha sido importante para la planeaci�n de los
experimentos realizados y comprende varios resultados que se mostrar�n m�s adelante.

Cada una de las bases de datos con im�genes empleadas
consiste en dos conjuntos fijos de im�genes, uno para entrenamiento y otro para
prueba. Estos conjuntos se seleccionaron aleatoriamente de la base de datos respectiva. Adicionalmente, las bases de datos descritas pueden
tener im�genes para prop�sitos especiales, las cuales se explican en
los experimentos que las ocupan. Para ciertos experimentos realizados
se utilizaron los conjuntos fijos de entrenamiento, mientras que para otros ambos conjuntos se seleccionaron aleatoriamente de entre
toda la base de datos respectiva. Todos los experimentos realizados con el
clasificador LIRA utilizaron las bases de datos A, B y D con excepci�n de los experimento sobre distorsiones y sobre conjunto ampliado de entrenamiento.

Los experimentos realizados, su justificaci�n, metodolog�a y resultados se abordan en las secciones siguientes.

\subsection{Experimentos preliminares}\label{ssec:LIRAexpPreliminares}
Las primeras pruebas con el clasificador LIRA han sido reportadas en \cite{Gengis2004}. Estas pruebas utilizaron la base de datos $\alpha$ descrita en la Secci�n \ref{ssec:BDI}. Se probaron varias combinaciones de par�metros para el clasificador considerando los mejores par�metros reportados en \cite{Kussul2004IVC}. En este grupo de pruebas no se utilizaron distorsiones para el conjunto de im�genes de entrenamiento. En la Tabla \ref{t:LIRAexpPreliminares} se muestran los resultados de este experimento. El mejor resultado obtenido en el experimento fue con una ventana LIRA de $15\times 15$ p�xeles, 175 000 neuronas en la capa $A$, cuatro neuronas ON y tres neuronas OFF en cada grupo, el par�metro $\eta $ del clasificador igual a 1.0 y el par�metro de entrenamiento $T_{E}$ igual a 0.15. Para estos par�metros el porcentaje de neuronas activas en la capa $A$ fue 0.164\%. Para esta prueba el porcentaje de reconocimiento correcto fue de 94\%. El mejor desempe�o obtenido para m�ltiples combinaciones de par�metros fue alcanzado con 40 ciclos de entrenamiento por lo que los mismos se utilizaron para todas las pruebas del experimento descrito.

\begin{table*}[!ht]
\caption[Experimento preliminar para la sintonizaci�n de par�metros LIRA.]{Experimento preliminar con nueve pruebas para la sintonizaci�n de par�metros LIRA con la base de datos $\alpha $.}
\label{t:LIRAexpPreliminares}\centering
\par
\resizebox{\textwidth}{!}{
\begin{tabular}{|c||c|c|c|c|c|c|c|c|}
\hline
No. de experimentos & 1 & 2 & \textbf{3} & 4 & 5 & 6 & 8 & 9 \\
\hline
\hline
Tama�o de la ventana ($w\cdot h)$ & $12\cdot 12$ & $15\cdot 15$ & $%
\mathbf{15\cdot 15}$ & $10\cdot 10$ & $13\cdot 13$ & $17\cdot 17$ & $15\cdot 15$ & $10\cdot 10$ \\ \hline
Constante LIRA ($\eta $) & 0.8 & 1.0 & \textbf{1.0} & 1.0 & 1.0
& 1.0 & 1.0 & 1.0 \\ \hline
Neuronas capa A ($N$) (miles) & $175$ & $175$ & $\mathbf{175}$ & $175$ & $%
175 $ & $175$ & $200$ & $200$ \\ \hline
Neuronas ON por grupo ($p$) & 3 & 3 & \textbf{4} & 4 & 4 & 5 & 4 & 4 \\ 
\hline
Neuronas OFF por grupo ($n$) & 4 & 4 & \textbf{3} & 3 & 3 & 3 & 3 & 3 \\ 
\hline
Neuronas activas (\%) & 0.06 & 0.09 & \textbf{0.16} & 0.12 & 0.13 & 0.08 & 0.17 & 0.15 \\ \hline
Porcentaje de reconocimiento (\%) & 68 & 89 & \textbf{94} & 93 & 85 & 88 & 89 & 89 \\ \hline
\end{tabular}
}
\end{table*}


Luego de m�ltiples pruebas preliminares y otras adicionales se determin� que el par�metro de entrenamiento $T_{E}$ igual a 0.15 es el que mejor rendimiento ofreci� para todos los casos y bases de datos por lo que se fijo al valor dado.

Los experimentos de las secciones siguientes se realizaron teniendo como base el mejor conjunto de par�metros obtenido en este experimento preliminar, es por ello que en lo sucesivo se les har� referencia a este conjunto de par�metros como par�metros base.

\subsection{Unicidad}\label{sec:LIRAexpUnicidad} 
Al inicio de esta secci�n se ha hecho menci�n de la caracter�stica �nica de cada construcci�n de un determinado clasificador LIRA, a�n con los mismos par�metros. Para estudiar cu�l es el comportamiento de tales clasificadores construidos as� se realiz� el experimento descrito a continuaci�n. Este experimento consisti� en una serie de pruebas con distintas construcciones LIRA utilizando los mismos par�metros y utilizando los mismos conjuntos fijos de entrenamiento y prueba, siendo entrenadas con el mismo n�mero de ciclos de entrenamiento. Estas consideraciones llevan a que la �nica variable del experimento es la estructura �nica de cada construcci�n LIRA. Este experimento se describe en primer lugar ya que para experimentos posteriores donde se utilizan distintas construcciones del clasificador LIRA adem�s de la variable que se pretenda estudiar se tendr� inevitablemente la variabilidad de la estructura particular de cada construcci�n LIRA. Por esta raz�n a este experimento se le llama de unicidad, derivado del hecho de que cada clasificador LIRA es �nico.

En la Tabla \ref{t:LIRAexpUnicidad} se muestran los resultados obtenidos para
10 pruebas realizadas con los par�metros base\footnote{Tanto por el clasificador LIRA como para el PCNC la ventana respectiva es cuadrada, por lo que el valor del par�metro $h$ es igual al de $w$. Por esta raz�n de aqu� en adelante se da s�lo el valor para $w$.}: $w=15$, $\eta=1.0$,
$N=175 000$, $p=4$ y $n=3$ y 40 ciclos de
entrenamiento. Adicionalmente se presenta el promedio de
reconocimiento obtenido y la desviaci�n est�ndar para mejor
comprensi�n de estos datos.

\begin{table*}[!ht]
\caption[Experimento de unicidad con LIRA.]{Resultados del experimento con 10 construcciones LIRA para cada una de las tres bases de datos. Cada una de las 10 corridas se hizo con id�nticos par�metros, mismos conjuntos fijos de entrenamiento y prueba, y mismo n�mero de ciclos de entrenamiento. Todos los resultados se dan en unidades porcentuales que indican el reconocimiento logrado.}
\label{t:LIRAexpUnicidad}
\centering
\par
\begin{tabular}{|r||c|c|c|c|c|c|c|c|c|c|c|c|}
\hline
LIRA: & 1 & 2 & 3 & 4 & 5 & 6 & 7 & 8 & 9 & 10 & $\bar{x}$ & $\sigma$\\
\hline
\hline
BD-A & 87 & 86 & 91 & 89 & 86 & 90 & 81 & 90 & 93 & 84 &87.7 & 3.14 \\
\hline
BD-B  & 93 & 90 & 90 & 92 & 91 & 91 & 92 & 91 & 93 & 91 &91.4 & 1.02 \\
\hline
BD-D  & 87 & 89 & 88 & 89 & 84 & 88 & 89 & 88 & 89 & 89 &88.0 & 1.48  \\
\hline
\end{tabular}
\end{table*}

...

%p
\begin{table*}[!ht]
\caption[Experimento con el par�metro $p$.]{Resultados de pruebas del clasificador LIRA con variaci�n del par�metro $p$ sobre las bases de datos A, B y D.}
\label{t:LIRAexpParamp}\centering
\par
\begin{tabular}{|r|c||c|c|c|c|c|}
\hline
Par�metro      & $p$&  2 & 3  & 4  & 5 & 6 \\
\hline
\hline
BD-A & \% & 85 & 88 & \textbf{89} & 83 & 86 \\
\hline
BD-B & \% & 79 & 69 & \textbf{94} & 82 & 85 \\
\hline
BD-D & \% & 86 & 85 & 85          & \textbf{89} & 86 \\
\hline
\end{tabular}
\end{table*}

%n
\begin{table*}[!ht]
\caption[Experimento con el par�metro $n$.]{Resultados de pruebas del clasificador LIRA con variaci�n del par�metro $n$ sobre las bases de datos A, B y D.}
\label{t:LIRAexpParamq}\centering
\par
\begin{tabular}{|r|c||c|c|c|c|c|c|}
\hline
Par�metro& $n$& 1  & 2 &    3 & 4 & 5 & 6\\
\hline
\hline
BD-A & \% & 35 & 78& \textbf{89} & 86& 83& 64\\
\hline
BD-B & \% & 73 & 82&  \textbf{94}& 83& 88& 80 \\
\hline
BD-D & \% & 85 & \textbf{86}&    85       & 80& 83& 78 \\  
\hline
\end{tabular}
\end{table*}

%pyq
\begin{table*}[!ht]
\caption[Experimento con los par�metro $p$ y $n$.]{Resultados de
  pruebas del clasificador LIRA con variaci�n de los par�metros $p$ y
  $n$ sobre las bases de datos A, B y D para distintos valores de $p+n$.}
\label{t:LIRAexpParampYn}\centering
\par
\begin{tabular}{|r||c|c|c|c|c|c|c|}
\hline
Par�metros $p$/$n, \; p+n=6$ & $1$/$5$& $2$/$4$& $3$/$3$& $4$/$2$& $5$/$1$& - & - \\
\hline
\hline
BD-A (\%): &  63         &  83             &   82        &   78        &    33       & - & -\\
\hline
BD-B (\%): &  82         &  88             &   91        &   92     &    84       & - & -\\
\hline
BD-D (\%): &  77         &  85             &   85        &   \textbf{87}&   80        & - & - \\
\hline
\hline
Par�metros $p$/$n, \; p+n=7$ & $1$/$6$& $2$/$5$& $3$/$4$& $4$/$3$& $5$/$2$& $6$/$1$ & -\\
\hline
\hline
BD-A (\%): &   33        &   73      &   \textbf{90} &     89      &    66     &    43  & -     \\
\hline
BD-B (\%): &  80         &   89      &   88        &\textbf{94}    &
91        &    87  & -\\
\hline
BD-D (\%): &  74         &   81      &   84        &       85      &
83        &    81  & -\\
\hline
\hline
Par�metros $p$/$n, \; p+n=8$ & $1$/$7$& $2$/$6$& $3$/$5$& $4$/$4$& $5$/$3$& $6$/$2$ & $7$/$1$\\
\hline
\hline
BD-A (\%): &   43        &   58        &   80        &   83        &   86        &   72 & 55   \\
\hline
BD-B (\%): &  78         &   80        &   87        &   83        &    86       & 85 & 82 \\
\hline
BD-D (\%): &  72         &   82        &   81        &   85        &\textbf{87} & 78 & 80 \\    
\hline
\end{tabular}
\end{table*}

El an�lisis de los datos obtenidos da muestra de la relaci�n de cada
uno de los par�metros del clasificador LIRA sobre su resultado. Se ha visto que los par�metros $w$ y $N$ influyen en el resultado para cada base de datos en forma mas o menos uniforme mientras que $p$, $n$ y $\eta$ no lo hacen.

Ahora bien, debido a la interdependencia que los par�metros juegan en
el desempe�o del clasificador LIRA, debe dudarse que los mejores par�metros obtenidos de forma individual garanticen que un conjunto formado por los mismos sea
el que mejor resultados proporcione. Para esto es necesario realizar otro
experimento en donde se prueben diversos conjuntos de par�metros que
tengan como base los resultados obtenidos tanto previamente como en esta
secci�n. Por lo tanto se han probado tres conjuntos de par�metros
sobre la base de datos A, el
primer conjunto es el de los par�metros obtenidos en la Secci�n \ref{ssec:LIRAexpPreliminares}, el segundo es el de los par�metros que dieron el mejor resultado en el experimento de par�metros independientes y el
tercer conjunto es el construido con los par�metros que fueron mejores
en los experimentos individuales. Para este experimento se han hecho
cinco distintas construcciones LIRA para cada conjunto de par�metros, para obtener el promedio de reconocimiento correcto para cada
uno en lugar de conformarse con un s�lo resultado, pues como se
explic� en la Secci�n \ref{sec:LIRAexpUnicidad} diversas construcciones
LIRA con id�nticos par�metros dan resultados con cierta variaci�n. N�tese que en la Tabla \ref{t:LIRAexpParamEta} se reporta el mejor resultado
obtenido en estos experimentos de sintonizaci�n. Este resultado es de 92\% de reconocimiento correcto. En la Tabla \ref{t:LIRAexpMejor} se resumen las pruebas realizadas para este experimento con las cinco corridas para cada uno de los 3 conjuntos de par�metros descritos, los resultados obtenidos y el promedio obtenido para cada caso.

\begin{table*}[!ht]
\caption[Mejor conjunto de par�metros LIRA.]{Resumen de resultados de pruebas para obtener el mejor conjunto de par�metros LIRA para la base de datos A. $\bar{x}$ es el promedio y $\sigma$ es la desviaci�n est�ndar.}
\label{t:LIRAexpMejor}\centering
\par
\resizebox{\textwidth}{!}{
\begin{tabular}{|l||c|c|c|c|c||c|c|c|c|c||c|c|}
\hline
Par�metros & \multicolumn{5}{|c|}{Valores} & \multicolumn{5}{|c|}{Construcciones/(\%)} & $\bar{x}$ & $\sigma$ \\
\hline
                  & $w$ &$\eta$&$N$   &$p$&$n$ & 1 & 2 &  3 &  4 &  5 & (\%) & (\%) \\
\hline
\hline
Base             & 15 & 1.0 & 175 000 & 4 & 3 & 81 & 89 & 91 & 92 & 90 & 88.6 & 3.93 \\
\hline
Mejor en pruebas & 15 & 0.9 & 175 000 & 4 & 3 & 85 & 91 & 87 & 88 & 91 & 88.4 & 2.33 \\
\hline
Mejores aislados & 15 & 0.9 & 200 000 & 3 & 4 & 78 & 85 & 89 & 85 & 90 & 85.4 & 4.22 \\
\hline
\end{tabular}
}
\end{table*}

...

Para el par�metro $D_{c}$ que es la distancia de correlaci�n del codificador del PCNC los resultados obtenidos se muestran en la Tabla \ref{t:PCNCexpParamDc}. Se tiene que para la base de datos A hubo un empate con $D_{c}$ igual a 2 y 5, mientras que para la base de datos B el mejor resultado fue con $D_{c}=6$ y para la base de datos D existi� tambi�n un empate para $D_{c}$ igual a 4 y 8. Estos resultados nos dicen, para la tarea particular que se trata, que la distancia de correlaci�n debe tener un valor peque�o menor que 10 pero mayor que uno.

%Dc
\begin{table*}[!ht]
\caption[Experimento con el par�metro $D_{c}$.]{Resultados de pruebas del PCNC con variaci�n del par�metro $D_{c}$ sobre las bases de datos A, B y D.}
\label{t:PCNCexpParamDc}\centering
\par
\begin{tabular}{|r|c||c|c|c|c|c|c|c|c|c|c|}
\hline
Par�metro & $D_{c}$ & 1  & 2 & 4 & 5 & 6 & 8 & 10 & 15 & 20 & 25 \\
\hline
\hline
BD-A      & \% & 70 & \textbf{93} & 91 & \textbf{93} & 92 & 90 & 90 & 88 & 87 & 74 \\ 
\hline
BD-B      & \% & 84 & 91 & 93 & 94 & \textbf{95} & 92 & 89 & 91 & 82 & 82 \\ 
\hline
BD-D      & \% & 77 & 72 & \textbf{85} & 82 & 83 & \textbf{85} & 78 & 82 & 72 & 76 \\ 
\hline
\end{tabular}
\end{table*}

El par�metro q es un factor del CDT que combina las distintas propiedades encontradas para una imagen determinada (Sec. \ref{sssec:CDT}). El resultado de las pruebas con este par�metro se muestran en la Tabla \ref{t:PCNCexpParamq}. El resultado de variar este par�metro sobre el resultado del PCNC es bastante irregular por lo que s�lo puede entenderse tomando en cuenta las posibles variaciones por el concepto de unicidad (Sec. \ref{sec:PCNCexpUnicidad}). As� se tiene que para la base de datos A el mejor resultado se tuvo para $q=5$, para la base de datos B para $q=10$ mientras que para la base de datos D el mejor resultado se obtuvo al no aplicar el CDT, esto es $q=0$.

%q
\begin{table*}[!ht]
\caption[Experimento con el par�metro $q$.]{Resultados de pruebas del PCNC con variaci�n del par�metro $q$ sobre las bases de datos A, B y D.}
\label{t:PCNCexpParamq}\centering
\par
\begin{tabular}{|r|c||c|c|c|c|c|c|c|c|c|c|c|c|c|c|}
\hline
Par�metro& $q$& 0  & 1 & 2  &  3 & 4 & 5 & 6 & 7  & 8  & 9  & 10 & 11 & 12 & 13 \\
\hline
\hline
BD-A & \%     & 90 & 92& 93 & 93& 92&\textbf{96}&93 & 94 & 87 & 88 & 86 & 91 & 93 & 93 \\
\hline
BD-B & \%     & 94 & 93&  94& 90& 94& 94 &92 & 92 & 89 & 90 &\textbf{97}& 95 & 93 & 93 \\
\hline
BD-D & \%     &\textbf{85}& 80&  82& 83 &80& 84& 80 & 82 & 78 & 80 & 80 & 79 & 83 & 82 \\  
\hline
\end{tabular}
\end{table*}

Dados los anteriores resultados tenemos que el mejor conjunto de par�metros para la base de datos A se dio para los par�metros base pero con $q=5$ (Tabla \ref{t:PCNCexpParamq}). Consid�rese tambi�n el conjunto de par�metros construido tomando los mejores par�metros que aisladamente dieron mejores resultados en los experimentos considerando adem�s su eficiencia. As�, tomando estos dos conjuntos de par�metros junto con los par�metros base se realiz� el siguiente experimento con cinco creaciones distintas para cada conjunto de par�metros, para de esta forma obtener resultados estad�sticamente confiables. En la Tabla \ref{t:PCNCexpMejor} se muestran estos resultados donde se ve que los mejores par�metros obtenidos aisladamente tambi�n constituyeron el mejor conjunto de par�metros. Adicionalmente es de notar que este conjunto de par�metros logr� la menor desviaci�n est�ndar en sus respuestas.

\begin{table*}[!ht]
\caption[Mejor conjunto de par�metros PCNC.]{Resumen de resultados de pruebas para obtener el mejor conjunto de par�metros PCNC para base de datos A. $\bar{x}$ es el promedio y $\sigma$ es la desviaci�n est�ndar.}
\label{t:PCNCexpMejor}
\centering
\par
\resizebox{\textwidth}{!}{
\begin{tabular}{|l||c|c|c|c|c|c|c|c||c|c|c|c|c||c|c|}
\hline
Par�metros: & \multicolumn{8}{|c|}{Par�metros} & \multicolumn{5}{|c|}{Construcciones/(\%)} & $\bar{x}$ & $\sigma$ \\
\hline
                 & $w$ &$p$ &$n$&$S$   &$N$    &$K$ &$D_{c}$ &$q$ & 1 & 2 &  3 &  4 &  5 & (\%) & (\%) \\
\hline
\hline
Base             & 10 & 5   & 4 &1000&300000& 20 &5 & 2 & 93 & 85 & 93 & 93 & 92 & 91.2 & 3.12 \\
\hline
Mejor en pruebas & 10 & 5   & 4 &1000&300000 &20 &5 & 5 & 95 & 90 & 86 & 90 & 88 & 89.8 & 2.99 \\
\hline
Mejores aislados & 12 & 4   & 3 &2500&200000& 20 &5 & 5 & 93 & 93 & 94 & 88 & 94 & 92.5 & 2.24 \\
\hline
\end{tabular}
}
\end{table*}

El mismo procedimiento descrito se utiliz� para obtener los mejores par�metros para el clasificador PCNC aplicado a las bases de datos B y D obteniendo un porcentaje de reconocimiento de 97\% para la base de datos B con los par�metros $w=10$, $p=4$, $n=3$, $S=1500$, $N=300 000$, $K=20$, $D_{c}=6$ y $q=10$. Para la base de datos D el mejor resultado obtenido fue de 91\% de reconocimiento con los par�metros $w=11$, $p=4$, $n=3$, $S=2000$, $N=400 000$, $K=15$, $D_{c}=4$ y $q=0$. Este �ltimo resultado con $q=0$ implica que este PCNC no utiliz� el CDT para lograr el mejor desempe�o.

\subsection{Distorsiones}\label{ssec:PCNCdistor}
Con el objetivo de mejorar la capacidad de reconocimiento del clasificador PCNC se realiz� un par de pruebas empleando distorsiones de las im�genes originales del conjunto de entrenamiento para ampliarlo. La prueba se hizo de la misma forma que para el clasificador LIRA (Sec. \ref{ssec:LIRAdistor}). En la Tabla \ref{t:PCNCexpDistorsiones} se resumen estas pruebas.

\begin{table*}[!ht]
\caption[Experimento con distorsiones con el PCNC.]{Experimento usando distorsiones para aumentar el conjunto de entrenamiento del PCNC aplicado sobre la base de datos A.}
\label{t:PCNCexpDistorsiones}\centering
\par
\begin{tabular}{|l||c|c|}
\hline
                       & \multicolumn{2}{|c|}{N�mero de prueba} \\
\hline
                       & 1 & 2 \\
\hline
\hline
N�mero de distorsiones & 6 & 12 \\
\hline
distorsiones horizontales & +1, -1 & +2, +1, -1, -2 \\
\hline
distorsiones verticales & +1, -1 & +2, +1, -1, -2 \\
\hline
distorsiones angulares & +1$^{\circ}$, -1$^{\circ}$ & +2$^{\circ}$, +1$^{\circ}$, -1$^{\circ}$, -2$^{\circ}$ \\
\hline
Ciclos de entrenamiento & 30 & 30 \\
\hline
Porcentaje de reconocimiento (\%)& 93 & 91 \\
\hline
\end{tabular}
\end{table*}

Los resultados obtenidos con las distorsiones son similares a los obtenidos para el clasificador LIRA, sin embargo considerando el mejor porcentaje de reconocimiento del PCNC las distorsiones no contribuyeron a mejorar el comportamiento del PCNC y si se considera adem�s el tiempo adicional requerido para crear las distorsiones y para el entrenamiento de �stas la conclusi�n es que su aplicaci�n no es conveniente para la tarea que nos ocupa. La explicaci�n a esto se da por el hecho que los clasificadores son entrenados para el reconocimiento de im�genes normalizadas y el agregar distorsiones s�lo distrae la memoria del clasificador hacia casos que se apartan de tal normalizaci�n. A medida que los clasificadores reconozcan una pieza ligeramente desviada de su estado normalizado (Sec. \ref{ssec:ImN}) con respuesta similar a su estado normalizado entonces no podr� aplicarse tal clasificador para la correcta identificaci�n del �ngulo de orientaci�n o posici�n cartesiana de tal imagen como se ver� m�s adelante en la Sec. \ref{sec:Localizacion}.

\subsection{Ciclos de entrenamiento}
Se realiz� la misma prueba que para LIRA con los ciclos de entrenamiento. Se crearon dos PCNCs con id�nticos par�metros para cada una de las base de datos utilizadas. En este experimento se tuvo una respuesta m�s r�pida para alcanzar el reconocimiento m�ximo respecto a LIRA. Por esta raz�n las pruebas se hicieron con pasos de 5 en lugar de 10 ciclos de entrenamiento. En la Tabla \ref{t:PCNCexpCiclosEnt} se muestran los resultados obtenidos. Para la base de datos A el reconocimiento m�ximo se alcanz� desde los 15 ciclos de entrenamiento, sin embargo se dio un resultado notable a solo 10 ciclos para la segunda prueba, pues se alcanz� un resultado de 93\% pero luego con ciclos adicionales cay� esta respuesta y no volvi� a repetirse. Para la base de datos B la respuesta fue uniforme y se tuvo a los 15 ciclos para la prueba 3 y a los 25 ciclos para la prueba 4. Para la base de datos D sucedi� algo similar pero con 25 y 30 ciclos para la primera y segunda de sus pruebas (pruebas 5 y 6 respectivamente). Se deduce que con 30 ciclos de entrenamiento se obtienen resultados correctos.

%t
\begin{table*}[!ht]
\caption[Experimento con diversos ciclos de entrenamiento.]{Experimento con diversos ciclos de entrenamiento sobre seis clasificadores PCNC. Para cada base de datos los par�metros son los mismos. Las pruebas se realizaron en las bases de datos A, B y D.}
\label{t:PCNCexpCiclosEnt}\centering
\par
\begin{tabular}{|r||c|c|c|c|c|c|}
\hline
                               & \multicolumn{6}{|c|}{No. de ciclos de entrenamiento} \\
\hline
N�mero de prueba:              &  5 & 10  & 15 & 20 & 25 & 30 \\
\hline
\hline
BD-A 1:                        & 66 & 88  &\textbf{94}& 94 & 94  & 94 \\
\hline
BD-A 2:                        & 76 & 93  &\textbf{90}& 90 & 90  & 90  \\
\hline
\hline
BD-B 3:                        & 69 & 90  &\textbf{96}& 96 & 96  & 96  \\
\hline
BD-B 4:                        & 67 & 89  & 94 &\textbf{97}& 97  & 97 \\
\hline
\hline
BD-D 5:                        & 39 & 59  & 69 & 81 &\textbf{88}& 88 \\
\hline
BD-D 6:                        & 51 & 61  & 64 & 79 &  84 &\textbf{86}\\
\hline
\end{tabular}
\end{table*}

\subsection{Aleatoriedad de los conjuntos para entrenamiento y prueba}
Una prueba muy importante es la de poder entrenar y probar el clasificador sobre conjuntos de entrenamiento y prueba variables. Por ello se realiz� un experimento con estos conjuntos tomados aleatoriamente de toda la base de datos en la misma proporci�n que se hizo originalmente, esto es, 50\% para las base de datos A y B y 72\% para la base de datos D para los conjuntos respectivos de entrenamiento y el resto para los conjuntos de prueba respectivamente. Para cada base de datos distinta se utiliz� un PCNC distinto, pero para cada una de las diez pruebas de cada base de datos se utiliz� el mismo PCNC. Los resultados obtenidos mostrados en la Tabla \ref{t:PCNCexpEyPaleatorios} dan cuenta de la estabilidad del PCNC en cuanto a una misma estructura particular. Los porcentajes para cada base de datos variaron poco, lo que se ve en la desviaci�n est�ndar obtenida para cada caso.

%aleatoreidad
\begin{table*}[!ht]
\caption[Experimento con conjuntos aleatorios PCNC.]{Resultados de pruebas con el PCNC con conjuntos de entrenamiento y prueba seleccionados aleatoriamente para las bases de datos A, B y D. Para cada base de datos se utiliz� un mismo clasificador.}
\label{t:PCNCexpEyPaleatorios}\centering
\par
\begin{tabular}{|r||c|c|c|c|c|c|c|c|c|c|c|c|}
\hline
Prueba No.: & 1 & 2 & 3 & 4 & 5 & 6 & 7 & 8 & 9 & 10 & $\bar{x}$&$\sigma$\\
\hline
\hline
BD-A: & 94 & 92 & 93 & 93 & 90 & 90 & 91 & 93 & 91 & 94&92.1&1.45\\
\hline
BD-B: & 96 & 96 & 97 & 94 & 95 & 96 & 95 & 97 & 95 & 95&95.6&0.92\\
\hline
BD-D: & 90 & 90 & 89 & 91 & 95 & 88 & 87 & 92 & 89 & 90&90.1&2.12 \\
\hline
\hline
\end{tabular}
\end{table*}

\subsection{Mejores clasificadores PCNC}
En esta secci�n se describen a detalle los resultados obtenidos con los mejores PCNC obtenidos para las bases de datos utilizadas. Primero que nada se muestra en la Tabla \ref{t:PCNCexpMejores} un resumen de los par�metros utilizados para estos clasificadores as� como la respuesta obtenida por ellos a detalle. Para todos ellos se utilizaron 30 ciclos de entrenamiento. En la Tabla se muestra adem�s de los par�metros empleados en cada base de datos y el resultado porcentual exacto obtenido en el reconocimiento, un desglose del n�mero de errores en cada caso por cada una de las clases utilizadas as� como la contribuci�n porcentual de cada clase en el error total de reconocimiento. Cons�ltense las Tablas \ref{t:caracteristicasBDs} y \ref{t:clasesEnBDs} para interpretar las clases para cada base de datos. T�mese en cuenta que los percentiles totales no suman 100\% debido a la aproximaci�n a dos d�gitos decimales.

\begin{table*}[!ht]
\caption[Resultado con los mejores clasificadores PCNC.]{Resumen de
  resultados de los mejores clasificadores PCNC para las bases de
  datos utilizadas. T, ciclos de entrenamiento. R, porcentaje de reconocimiento.}
\label{t:PCNCexpMejores}\centering
\par
\resizebox{\textwidth}{!}{
\begin{tabular}{|p{1.2cm}||c|c|c|c|c|c|c|c||c|c||c|c|c|c|c|c|c|c|}
\hline
Base de datos: & \multicolumn{8}{|c|}{Par�metros} & T & R & \multicolumn{8}{|c|}{Errores por clase/(\%)} \\
\hline
                 & $w$ &$p$ &$n$&$S$   &$N$  &$K$ &$D_{c}$ &$q$ & & & 1 & 2 & 3 &  4 & 5 & 6 & 7 & 8 \\
\hline
\hline
A                & 10 & 5   & 4 &1000&300000& 20 &5 & 5 & 30 & 96.87  &4/2.50 & 0 & 0 & 0 & 0 & 0 & 1/0.63 & 0 \\
\hline
B                & 10 & 4   & 3 &1500&300000 &20 &6 & 10& 30 & 97.80  &1/0.37 &1/0.37& 0 &3/1.10& 0 &1/0.37 & 0 & - \\ 
\hline
C                & 11 & 4   & 3 &2000&400000& 15 &4 & 0 & 30 & 91.43  &6/5.71 & 0 & 1/0.95 & 1/0.95&1/0.95& 0 & 0 & - \\
\hline
\end{tabular}
}
\end{table*}

\subsubsection{Base de datos A}
El PCNC logr� obtener un porcentaje de error para la base de datos A mayor a 96\%. Solo 5 im�genes de un total de 160 no fueron reconocidas correctamente. Estas im�genes se repartieron en s�lo dos clases (base de tubito y tornillo cabeza c�nica) mientras que para las otras seis clases el reconocimiento fue del 100\% (cono, eje de rotor, no pieza central, terminal de cable, tornillo allen y tornillo cabeza redonda). En la Fig. \ref{fig:PCNCreconocidas-BD-A} se presentan dos ejemplos de cada clase de entre las im�genes reconocidas y en la Fig. \ref{fig:PCNCmalReconocidas-BD-A} se presentan todas las im�genes que no se reconocieron correctamente.

%BD-A buenas
\begin{figure}
[!ht]
\centering
 %  \setcounter{subfiggroup}{1}
\subfloat[]{\label{fig:PCNC-BD-Abiena}\includegraphics[width=0.14\textwidth]{figuras/BDIA/PCNC/bien_selec/base_de_tubito_0035.png}}
\subfloat[]{\label{fig:PCNC-BD-Abienb}\includegraphics[width=0.14\textwidth]{figuras/BDIA/PCNC/bien_selec/base_de_tubito_0026.png}}
\subfloat[]{\label{fig:PCNC-BD-Abienc}\includegraphics[width=0.14\textwidth]{figuras/BDIA/PCNC/bien_selec/eje_de_rotor_0022.png}}
\subfloat[]{\label{fig:PCNC-BD-Abiend}\includegraphics[width=0.14\textwidth]{figuras/BDIA/PCNC/bien_selec/eje_de_rotor_0037.png}}
\subfloat[]{\label{fig:PCNC-BD-Abiene}\includegraphics[width=0.14\textwidth]{figuras/BDIA/PCNC/bien_selec/cono_0021.png}}
\subfloat[]{\label{fig:PCNC-BD-Abienf}\includegraphics[width=0.14\textwidth]{figuras/BDIA/PCNC/bien_selec/cono_0038.png}}
\subfloat[]{\label{fig:PCNC-BD-Abieng}\includegraphics[width=0.14\textwidth]{figuras/BDIA/PCNC/bien_selec/no_pieza_central_0037.png}}
\\
\subfloat[]{\label{fig:PCNC-BD-Abienh}\includegraphics[width=0.14\textwidth]{figuras/BDIA/PCNC/bien_selec/no_pieza_central_0023.png}}
\subfloat[]{\label{fig:PCNC-BD-Abieni}\includegraphics[width=0.14\textwidth]{figuras/BDIA/PCNC/bien_selec/terminal_de_cable_0033.png}}
\subfloat[]{\label{fig:PCNC-BD-Abienj}\includegraphics[width=0.14\textwidth]{figuras/BDIA/PCNC/bien_selec/terminal_de_cable_0034.png}}
\subfloat[]{\label{fig:PCNC-BD-Abienk}\includegraphics[width=0.14\textwidth]{figuras/BDIA/PCNC/bien_selec/tornillo_allen_0022.png}}
\subfloat[]{\label{fig:PCNC-BD-Abienl}\includegraphics[width=0.14\textwidth]{figuras/BDIA/PCNC/bien_selec/tornillo_allen_0036.png}}
\subfloat[]{\label{fig:PCNC-BD-Abienm}\includegraphics[width=0.14\textwidth]{figuras/BDIA/PCNC/bien_selec/tornillo_cabeza_conica_0027.png}}
\subfloat[]{\label{fig:PCNC-BD-Abienn}\includegraphics[width=0.14\textwidth]{figuras/BDIA/PCNC/bien_selec/tornillo_cabeza_conica_0035.png}}
\\
\subfloat[]{\label{fig:PCNC-BD-Abieno}\includegraphics[width=0.14\textwidth]{figuras/BDIA/PCNC/bien_selec/tornillo_cabeza_redonda_0030.png}}
\subfloat[]{\label{fig:PCNC-BD-Abienp}\includegraphics[width=0.14\textwidth]{figuras/BDIA/PCNC/bien_selec/tornillo_cabeza_redonda_0032.png}}
\caption[Im�genes reconocidas en la BD-A por el PCNC.]{Ejemplos de im�genes correctamente reconocidas con el mejor PCNC para la base de datos A. Se muestran dos ejemplos por cada clase.}%
\label{fig:PCNCreconocidas-BD-A}%
\end{figure}

...

\subsubsection{Base de datos B}
En la base de datos B existi� el mejor resultado obtenido en el presente trabajo. Este resultado fue de 97.80\% de reconocimiento correcto. Dos ejemplos de cada una de las clases tomadas del grupo de 267 im�genes bien reconocidas se muestran en la Fig. \ref{fig:PCNCreconocidas-BD-B}. Las im�genes mal reconocidas se distribuyeron en cuatro de las siete clases de la base de datos B de la siguiente forma: un tornillo plano, tres tornillos allen grandes, una no pieza central y una arandela. Las clases tornillo allen chico, tornillo gota y tuerca se reconocieron sin errores. En la Fig. \ref{fig:PCNCmalReconocidas-BD-B} se muestran las seis im�genes mal reconocidas.

...


\subsection{Prueba con conjunto de entrenamiento ampliado}\label{ssec:PCNCentrenarAmpliado}
La misma prueba que se aplic� para LIRA con un conjunto de entrenamiento ampliado de la base de datos D, se aplic� para el PCNC. Esta ampliaci�n consiste en a�adir las im�genes extras descritas en la Secci�n \ref{ssec:LIRAentrenarAmpliado}. El resultado de entrenar al PCNC con 537 im�genes agregadas al conjunto de entrenamiento result� en un porcentaje de reconocimiento sobre el conjunto de prueba de 93\%. Los errores fueron cinco, distribuidos en una no pieza central, dos tornillos allen grandes, un tornillo gota y un tornillo plano.

\section{Comparaci�n de clasificadores}
El prop�sito de esta secci�n es comparar los resultados obtenidos de ambos clasificadores en las distintas pruebas realizadas con el objetivo de facilitar la elecci�n de uno u otro atendiendo a la aplicaci�n que se requiera.

\subsection{Tiempos}
El porcentaje de reconocimiento m�ximo que sobre una base de
datos de im�genes logre un clasificador es el par�metro principal para
medir su eficiencia. Sin embargo el consumo de recursos que se tenga
es un factor muy importante tambi�n y en especial el tiempo. Es por
ello que se muestran y comparan los tiempos empleados por ambos
clasificadores utilizados para una misma base de datos y en un mismo
equipo de c�mputo. Con esto se tiene una clara y objetiva comparaci�n
entre los clasificadores, considerando adem�s el resultado obtenido en
cuanto al porcentaje de reconocimiento. Estos datos son
presentados en la Tabla \ref{t:tiempos}.

\begin{table*}[!ht]
\caption[Tiempos empleados por los clasificadores.]{Tiempos empleados por los clasificadores LIRA y PCNC sobre la base de datos A.}
\label{t:tiempos}
\centering
\resizebox{\textwidth}{!}{
\begin{tabular}{|l|c||c|c|c|c|c|c|}
\hline
        &                        & \multicolumn{6}{|c|}{Tiempos por tarea}           \\
\hline
        &                        & Creaci�n &\multicolumn{2}{|c|}{Entrenamiento (160 im�genes)} & Prueba & Guardar & Reconocer \\
\hline
Clasificador &Reconocimiento (\%) & o carga  & codificaci�n  & entrenamiento & (160 im�genes) & en disco & una imagen\\
\hline
\hline
LIRA         &   93\%             &  3.08s   & 70s         & 154s          & 110s    &  0.65s  & 0.687s\\
\hline
PCNC         &   97\%             &  1.25s   & 198s          & 148s          & 222s  &  0.67s & 1.387s \\   
\hline
\end{tabular}
}
\end{table*}

Todos estos tiempos dependen de los par�metros de los clasificadores, como se explic� el par�metro $N$ para el clasificador LIRA y los par�metros $S$ y $N$ para el PCNC son los que m�s impactan en los tiempos de ejecuci�n de las diversas tareas de estos clasificadores. Los resultados presentados en la tabla son para el mejor clasificador para cada uno de ellos para la base de datos A. Tenemos que los tiempos de creaci�n, carga y salvado de estos clasificadores es peque�o y considerando que en un trabajo continuo de reconocimiento s�lo ser�n cargados una vez entonces este tiempo es m�nimo y no impacta el desempe�o de ninguno de los clasificadores. Lo mismo aplica para el tiempo de guardado. Para el caso del entrenamiento dividido en el tiempo de codificaci�n y de entrenamiento de los c�digos, se tiene que para el PCNC el tiempo de codificaci�n es casi de tres veces el respectivo para LIRA, mientras que los tiempos de entrenamiento de c�digos es muy similar. Si se considera que el entrenamiento de una determinada base de datos a utilizar se hace una sola vez, entonces, al menos para este caso, la ventaja de cuatro puntos porcentuales m�s de reconocimiento que tiene el PCNC pagar� el costo en el tiempo del entrenamiento. Respecto al tiempo de prueba, el PCNC requiere un tiempo doble que LIRA. El mismo argumento que para el tiempo de entrenamiento puede esgrimirse para los tiempos de prueba. Sin embargo, los tiempos de reconocimiento de una imagen particular tienen la misma proporci�n entre los clasificadores. Es este tiempo el factor clave que dependiendo de la aplicaci�n y uso particular que se requiera debe ser tomado en cuenta junto con el porcentaje de reconocimiento de cada clasificador para escoger uno de ellos. En general si la tarea no es cr�tica en tiempo debiera emplearse el PCNC debido a su poder superior de reconocimiento, mientras que si la tarea particular exige tiempo cr�ticos y a su vez puede tolerar m�s errores entonces el clasificador LIRA debe ser utilizado.

\subsection{Estabilidad en la creaci�n de estructuras}
En los experimentos de unicidad de LIRA y del PCNC se crearon y probaron diez clasificadores para cada uno con id�nticos par�metros, ciclos de entrenamiento y conjuntos de entrenamiento y prueba. De estos experimentos la variable resultante de importancia es la desviaci�n est�ndar. T�ngase en cuenta que la desviaci�n est�ndar es una medida de la variaci�n de los resultados respecto a su promedio, por lo que a menor valor mejor confiabilidad se tiene en los resultados para un conjunto de par�metros dada a la hora de construir un clasificador con ellos. Los resultados presentados en las Tablas \ref{t:LIRAexpUnicidad} y \ref{t:PCNCexpUnicidad} se resumen y comparan en la Tabla \ref{t:comparaUnicidad}. Se tiene que LIRA fue superior en estabilidad para las bases de datos B y D, mientras que para la base de datos A tuvo una estabilidad menor que el PCNC. En base a estos resultados no puede declararse ning�n ganador entre los dos tipos de clasificadores probados en cuanto a esta caracter�stica.

\begin{table*}[!ht]
\caption[Comparaci�n de unicidad entre los clasificadores.]{Comparaci�n en la desviaci�n est�ndar obtenida entre los clasificadores LIRA y PCNC sobre las tres bases de datos utilizadas.}
\label{t:comparaUnicidad}
\centering
\begin{tabular}{|c||c|c|}
\hline
              & \multicolumn{2}{|c|}{Clasificador}\\
\hline
Base de datos:& LIRA   &  PCNC \\
\hline
\hline
A             & 3.14\%&2.24\%\\
\hline
B             & 1.02\%&1.58\%\\
\hline
D             & 1.48\%&3.75\%\\
\hline
\end{tabular}
\end{table*}

\subsection{Par�metros}
Para ambos clasificadores se hicieron una serie de experimentos variando un s�lo par�metro a la vez para estudiar la forma en que var�an los resultados obtenidos y para encontrar el mejor clasificador para cada base de datos. Sin embargo para cada una de estas pruebas es necesario construir un nuevo clasificador por lo que siempre se tiene a�adido al resultado obtenido la incertidumbre dada por la unicidad. Dado el argumento anterior y el estudio de las estructuras LIRA y PCNC se concluye que la b�squeda de par�metros para ambos clasificadores utilizados requiere de m�ltiples pruebas con cada base de datos a utilizar.

\subsection{Ciclos de entrenamiento}
De los experimentos hechos con los clasificadores para encontrar el mejor n�mero de ciclos de entrenamiento se tiene que el clasificador LIRA requiri� entre 30 y 70 ciclos de entrenamiento para alcanzar un valor estable mientras que el PCNC s�lo requiri� entre 15 y 30 ciclos. Esta diferencia no influye significativamente en el rendimiento total de los clasificadores pues ambos utilizan codificaci�n del conjunto de entrenamiento para evitar codificar cada nuevo ciclo de entrenamiento. Siendo el tiempo de entrenar cada ciclo adicional para ambos clasificadores muy similar como puede verse en la tabla \ref{t:tiempos}

\subsection{Confiabilidad}
Una vez construido un clasificador debe conocerse que tan confiable y estable es �ste para reconocer objetos. Para analizar esta propiedad sobre las bases de datos y conocer en qu� medida varia una construcci�n �nica de un determinado clasificador se realizaron los experimentos con conjuntos aleatorios de entrenamiento y prueba. El resumen y la comparaci�n de estos experimentos se tiene en la Tabla \ref{t:comparaConfiabilidad}

Se tiene que el PCNC fue en todos los casos m�s confiable que el clasificador LIRA. Para las base de datos A y B super� por m�s del doble a LIRA. Este resultado es de lo m�s importante por que da cuenta de la capacidad de los clasificadores de ser entrenados y probados con im�genes aleatoriamente seleccionadas.

\begin{table*}[!ht]
\caption[Confiabilidad de los clasificadores.]{Resumen de los experimentos de confiabilidad para los clasificadores LIRA y PCNC sobre las tres bases de datos utilizadas. Los resultados se expresan como la desviaci�n est�ndar obtenida sobre las diez pruebas realizadas para cada base de datos y cada clasificador.}
\label{t:comparaConfiabilidad}
\centering
\begin{tabular}{|c||c|c|}
\hline
              & \multicolumn{2}{|c|}{Clasificador}\\
\hline
Base de datos:& LIRA   &  PCNC \\
\hline
\hline
A             & 3.40\%&1.45\%\\
\hline
B             & 2.15\%&0.92\%\\
\hline
D             & 3.37\%&2.12\%\\
\hline
\end{tabular}
\end{table*}

\subsection{Resultados}
Luego de diversos experimentos se encontraron los mejores par�metros para cada clasificador y para cada base de datos. El desempe�o del PCNC fue superior al clasificador LIRA para todas las base de datos utilizadas. S�lo en la base de datos D este resultado fue casi igual. Un resumen de estos resultados se presenta en la Tabla \ref{t:comparaMejores}.

\begin{table*}[!ht]
\caption[Resultados de los clasificadores sobre las bases de datos.]{Resultados obtenidos por los clasificadores LIRA y PCNC sobre las bases de datos A, B y D. Los resultados expresan el porcentaje de reconocimiento sobre la base de datos respectiva.}
\label{t:comparaMejores}
\centering
\begin{tabular}{|c||c|c|}
\hline
              & \multicolumn{2}{|c|}{Clasificador}\\
\hline
Base de datos:& LIRA   &  PCNC \\
\hline
\hline
A             & 93.75\%&96.87\%\\
\hline
B             & 94.14\%&97.80\%\\
\hline
D             & 90.47\%&91.43\%\\
\hline
\end{tabular}
\end{table*}

Del an�lisis y comparaci�n de los resultados obtenidos por los mejores clasificadores para las bases de datos tenemos que para la base de datos A ambos clasificadores reconocieron el 100\% de las clases 2 a 6 (Tabla \ref{t:LIRAexpMejores}), es decir, las clases cono, eje de rotor, no pieza central, terminal de cable y tornillo allen. Para la base de datos D las clases 3 y 7 (tornillo allen chico y tuerca) fueron tambi�n reconocidas sin errores por ambos clasificadores. En cuanto a las base de datos B y D que est�n construidas con el mismo tipo de piezas y el mismo n�mero de clases pero de complejidad distinta, el clasificador LIRA no tuvo errores para la clase 4 (tornillo allen grande) mientras que el PCNC no tuvo errores para ninguna de estas dos bases de datos para la clase 3 (tornillo allen chico).


Comparando los errores obtenidos por cada clasificador para cada una de las base de datos utilizadas tenemos que las cinco im�genes mal reconocidas con el PCNC para la base de datos A se encuentran dentro de las diez mal reconocidas por LIRA para esta misma base de datos. Es decir, estas cinco im�genes no pudieron ser reconocidas por ninguno de los clasificadores (Fig. \ref{fig:PCNCmalReconocidas-BD-A}).

Para la base de datos B la comparaci�n da resultados distintos ya que s�lo una imagen no pudo ser reconocida por ambos clasificadores Fig. \ref{fig:LIRAmalReconocidas-BD-B} \ref{fig:PCNC-BD-Bmalb}. 

Por �ltimo, para la base de datos D cuatro im�genes no pudieron ser reconocidas por ninguno de las clasificadores. Estas im�genes son las mostradas en las Figs.:\ref{fig:LIRAmalReconocidas-BD-D} \ref{fig:LIRA-BD-Dmalc}, \ref{fig:LIRA-BD-Dmalh}, \ref{fig:LIRA-BD-Dmali}, \ref{fig:LIRA-BD-Dmalj} que corresponden a las Figs: \ref{fig:PCNCmalReconocidas-BD-D} \ref{fig:PCNC-BD-Dmalb}, \ref{fig:PCNC-BD-Dmala}, \ref{fig:PCNC-BD-Dmalf} y \ref{fig:PCNC-BD-Dmale} respectivamente.


Por otra parte ambos clasificadores no tuvieron problema en reconocer las piezas que ubic�ndose cerca del extremo de la escena\footnote{Se entiende por escena la imagen original de la cu�l se recortaron las im�genes para formar las bases de datos o sobre la cu�l se pretende localizar una pieza determinada.} poseen una parte blanca que rellena la falta de imagen original. Ejemplos de estas im�genes bien reconocidas se tienen en la Fig. \ref{fig:LIRAreconocidas-BD-A} \ref{fig:PCNC-BD-Abienb}, Fig. \ref{fig:LIRAreconocidas-BD-B} \ref{fig:PCNC-BD-Bbienn}, Fig. \ref{fig:PCNCreconocidas-BD-A} \ref{fig:PCNC-BD-Abienn} y Fig. \ref{fig:PCNCreconocidas-BD-B} \ref{fig:PCNC-BD-Bbieni}.

Otra cualidad muy importante de ambos clasificadores ha sido poder reconocer piezas parcialmente obstruidas como puede verse en las Figs. \ref{fig:LIRAreconocidas-BD-D} \ref{fig:PCNC-BD-Dbiena}, \ref{fig:PCNC-BD-Dbienb} y Fig: \ref{fig:PCNCreconocidas-BD-D} \ref{fig:PCNC-BD-Dbienb}.

\section{B�squeda y reconocimiento de posici�n}\label{sec:Localizacion}
Uno de los objetivos del presente trabajo es la creaci�n de un sistema autom�tico de localizaci�n de piezas o SVT (Sec. \ref{sec:SVT}). En este punto del trabajo ya se han expuesto dos m�todos de identificaci�n de piezas constituidos por los clasificadores LIRA y PCNC. Cualquiera de estos clasificadores ha demostrado, en los experimentos expuestos en las secciones anteriores, resultados suficientemente buenos por lo que cualquiera de ellos puede ser usado como base en el m�todo de localizaci�n de piezas. Una vez que el clasificador seleccionado es entrenado con una base de datos particular que contiene im�genes de las piezas con la que se desea trabajar, puede aplicarse el m�todo de localizaci�n de piezas sobre una imagen o escena particular. Dicho m�todo se ha expuesto en la Secci�n \ref{sec:localizacion}.

Como ejemplo se tom� el mejor clasificador LIRA para la base de datos A entrenado con las siete clases de esta base de datos y se realizaron distintas b�squedas para localizar una determinada pieza en alguna imagen dada.

En la Fig. \ref{fig:goodfoundworkpieces} mostramos dos ejemplos de im�genes reconocidas. Una de las im�genes contiene dos conos reconocidos (conos 1, 2 en la Fig. \ref{fig:goodfoundworkpieces}(a). En la Fig. \ref{fig:goodfoundworkpieces} (b) la imagen contiene tres terminales de cable reconocidas (1, 2, 3).

\begin{figure}
[!h]
\begin{center}
\includegraphics[
natheight=1.030900in,
natwidth=4.614600in,
height=3.6832in,
width=1.8in
]%
{figuras/goodFoundWorkPieces.jpg}%
\caption{Localizaci�n y reconocimiento de piezas.}%
\label{fig:goodfoundworkpieces}%
\end{center}
\end{figure}

El peque�o cuadro blanco en la figura representa el lugar donde la pieza
solicitada ha sido reconocida y el cuadro grande representa la ventana correspondiente de
b�squeda asociada a esta posici�n.

Al pie de las im�genes puede leerse las coordenadas de la posici�n y la
orientaci�n encontradas por el sistema para cada una de las piezas
reconocidas. 

El clasificador neuronal es un m�todo flexible, esto significa que el
sistema reconoce una pieza fuera de su centro, pero suficientemente cerca
del cuerpo de la pieza.

Algunas veces el sistema puede reconocer una misma pieza varias veces. En la
Fig. \ref{fig:redundantfoundworkpieces}(a) se muestra un reconocimiento m�ltiple de una pieza (casos 1 y 3). 

%\twocolumn

\begin{figure}
[!h]
\begin{center}
\includegraphics[
natheight=1.030900in,
natwidth=4.614600in,
height=3.6832in,
width=1.8in
]%
{figuras/redundantFoundWorkPieces.jpg}%
\caption{Localizaci�n y reconocimiento de piezas con redundancia.}%
\label{fig:redundantfoundworkpieces}%
\end{center}
\end{figure}

Esto sucede por que el sistema puede encontrar una pieza lejos de su centro, por ejemplo las marcas 2 y 3 en la Fig \ref{fig:redundantfoundworkpieces}(a) y
1, 7 en la Figura \ref{fig:redundantfoundworkpieces}(b). 

En otros casos, esto pasa porque los par�metros de b�squeda ($\Delta x$, $\Delta y$, $\Delta \theta $) tienen valores muy grandes, por ejemplo la marca 1 en la Fig \ref{fig:redundantfoundworkpieces}(a) y las marcas 2, 4, 8 y 9 en la Figura \ref{fig:redundantfoundworkpieces}(b).

La marca 4 de la Fig. \ref{fig:redundantfoundworkpieces}(a) y la marca 6 de la Fig. \ref{fig:redundantfoundworkpieces}(b) tienen suficiente precisi�n para el trabajo de manipulaci�n.

Dos ejemplos de im�genes con piezas que se tocan una a otra se presentan en la Fig. \ref{touchFoundWorkPieces}. En la Fig. \ref{touchFoundWorkPieces}(a) el sistema encontr� tres piezas (obs�rvese que el punto de localizaci�n est� lejos del centro de estas piezas). En la Fig. \ref{touchFoundWorkPieces}(b) el sistema presenta redundancia. En este caso el sistema puede filtrar los resultados considerando la mejor respuesta obtenida en la salida del clasificador LIRA, lo cual significa que la pieza con las marcas 2 y 3 se considerar� reconocida en la marca n�mero 2 que est� mucho mejor posicionada con respecto al centro de la pieza.

\begin{figure}
[!ht]
\begin{center}
\includegraphics[width=2.3298in]
{figuras/touchFoundWorkPieces.jpg}
\caption{Localizaci�n y reconocimiento de piezas que se tocan.}
\label{touchFoundWorkPieces}%
\end{center}
\end{figure}

Los resultados obtenidos son buenos para permitir la manipulaci�n, sin embargo el m�todo de b�squeda debe ser mejorado para incrementar la precisi�n de la localizaci�n y del �ngulo de la pieza a ser reconocida.

De los resultados presentados se tiene que en el sistema desarrollado existen dos problemas espec�ficos de reconocimiento que son: 

\begin{enumerate}
\item La redundancia en el reconocimiento y 
\item La precisi�n del localizador de piezas.
\end{enumerate}

Respecto a la redundancia en el reconocimiento se tiene que esto sucede debido a que los clasificadores utilizados tienen suficiente flexibilidad como para reconocer una pieza a�n lejos de su centro. Esto en primera instancia al momento de la localizaci�n representa una ventaja, sin embargo cuando se quiere obtener la localizaci�n exacta de la pieza entonces la misma caracter�stica representa un problema. Una forma de resolver este problema es comparando las respuestas neuronales de los clasificadores de la salida ganadora y seleccionando la que mayor valor tenga, la cu�l ser� a su vez el punto m�s cercano al centro de la pieza. Sin embargo esta situaci�n resta eficiencia al m�todo de localizaci�n por lo que m�s estudios deben hacerse al respecto para reducir este problema.

Con respecto a la precisi�n del localizador, �sta depende tanto de la estructura particular del clasificador que se utilice, como de su entrenamiento as� como de los par�metros propios del localizador. Si los par�metros que definen el avance de la ventana de b�squeda se reducen la precisi�n de b�squeda aumentar� sin embargo mayor tiempo de c�mputo ser� necesario; por el contrario, si estos par�metros se aumentan la precisi�n baja y el tiempo aumenta. Dado lo anterior es importante encontrar los valores �ptimos para los par�metros del localizador. Adem�s de lo anterior pueden introducirse mejoras al m�todo de localizaci�n como la realizaci�n de una b�squeda fina una vez que se ha localizada una regi�n donde exista una pieza de inter�s.

Considerando los resultados de los experimentos presentados con el localizador as� como los dos problemas que en ocasiones se presentan tenemos que el sistema es un excelente paso inicial para la localizaci�n e identificaci�n de piezas; sin embargo m�s investigaci�n y desarrollo debe de hacerse en esta �rea particular para mejorar las prestaciones del sistema de localizaci�n.

%%% Local Variables: 
%%% mode: latex
%%% TeX-master: "tesis"
%%% End: 


\chapter*{\addcontentsline{toc}{chapter}{Conclusiones}Conclusiones}
En el Laboratorio de Micromec�nica y Mecatr�nica se desarrolla microequipo de bajo costo para producir dispositivos micromec�nicas. Una caracter�stica que permite el bajo costo es la utilizaci�n de algoritmos adaptivos para compensar la mayor precisi�n respecto a los equipos de alto costo. Entre los algoritmos adaptivos destacan los basados en visi�n por computadora.

...


Para la implementaci�n de los dos clasificadores neuronales, para el localizador y para la creaci�n de las bases de datos de las im�genes se desarrollaron diversos m�dulos de software. Todo este software constituy� una herramienta indispensable para la investigaci�n y los experimentos realizados. El software se desarroll� mediante un lenguaje de Programaci�n Orientado a Objetos, C++ est�ndar, desde el sistema operativo GNU/Linux.

%%% Local Variables: 
%%% mode: latex
%%% TeX-master: "tesis"
%%% End: 

\include{futuro}

\part*{\addcontentsline{toc}{part}{Ap�ndices}Ap�ndices}

% Adjustments headers
\fancyhead[RO]{\leftmark}
\fancyhead[EL]{\emph{Ap�ndice \thechapter}}

%%%%%%%%%%%%%
\appendix
%%%%%%%%%%%%%
\chapter{Software}
Con el objetivo de contar con una herramienta para la investigaci�n y
los experimentos sobre los clasificadores neuronales LIRA y PCNC, as� como para el localizador y la creaci�n de las bases de datos, se crearon varios paquetes de software. El software se desarroll� mediante un lenguaje de Programaci�n Orientado a Objetos (POO). Los lenguajes orientados a
objetos permiten contar con m�dulos de software reutilizables,
flexibles y de f�cil mantenimiento. Esto
hace posible que la investigaci�n y desarrollo sean m�s eficientes en lo que a software se refieren
y pone a disposici�n inmediata aquellos m�dulos de software que han
sido probados con buenos resultados ya sea para un sistema completo,
un sistema alternativo o para realizar otro tipo de experimentos.

El software desarrollado en primera instancia se compone de cuatro m�dulos: Optik, RNA,
Localizador e Interfaz. El m�dulo Optik se encarga de la creaci�n de
bases de datos de im�genes, el m�dulo RNA es la implementaci�n del
clasificador neuronal LIRA, el m�dulo Localizador busca una pieza
determinada en una imagen y la Interfaz es el m�dulo encargado de la
intercomunicaci�n entre los dem�s m�dulos as� como con el usuario. En
la Fig. \ref{OptikRNADiagramaABloques} se muestra un diagrama a bloques de estos m�dulos as� como
las interconexiones entre �stos y los archivos que se manejan, tambi�n
se listan las funciones disponibles para el usuario. Se describen a
continuaci�n con m�s detalle cada uno de los m�dulos en los se har�
referencia continua a la figura mencionada.

\begin{figure}[h]
\begin{center}
\includegraphics[
%natheight=1.760800in,
%natwidth=2.603900in,
%height=1.2756in,
width=\textwidth
]{figuras/OptikRNADiagramaABloquesV2_1.png}
\caption[Diagrama a bloques del software OptikRNA.]{Diagrama a bloques del software OptikRNA. Se muestran los cuatro m�dulos que lo componen, los archivos utilizados y sus
   principales funciones as� como las intercomunicaciones entre todos
   estos elementos.}\label{OptikRNADiagramaABloques}
\end{center}
\end{figure}

El software desarrollado posee muchas ventajas de la programaci�n orientada a objetos (POO). Se utiliz� el lenguaje de programaci�n C++. En un inicio se us� el Ambiente integral de desarrollo Borland$^{MR}$ y con ello algunos objetos predefinidos por Borland$^{MR}$ fueron utilizados para la manipulaci�n de im�genes y la creaci�n de las interfaces gr�ficas.

M�s tarde se decidi� trasladar el c�digo existente a C++ est�ndar por lo que se abandon� el uso de dicho ambiente de desarrollo. El uso del C++ est�ndar posibilita que el c�digo pueda ser trasladado f�cilmente a cualquier plataforma y a sistemas embebidos, lo anterior es especialmente importante para nuestro proyecto. Adem�s, los costos de desarrollo se reducen al no depender de software comercial haciendo que el sistema sea m�s econ�mico, caracter�stica que es una de las pautas m�s importantes en todo el trabajo desarrollado. Bajo esta nueva pauta de trabajo se desarroll� tambi�n el software PCNC.

La aplicaci�n Optik est� siendo desarrollado para GNU/Linux [REF] y las dem�s aplicaciones para el clasificador LIRA y el PCNC se desarrollaron en C++ est�ndar en este mismo sistema operativo. Tambi�n el localizador de piezas. Si bien no se les desarroll� interfaz gr�fica, esto se hizo premeditadamente debido a las siguientes razones: 

\begin{enumerate} 
\item Las buenas pr�cticas de programaci�n moderna ense�an que la interfaz gr�fica debe separarse totalmente de la implementaci�n del software particular. 
\item El poder que ofrece el hecho de poder correr el software desde la l�nea de comandos cuyo ejemplo m�s pr�ctico se da en los archivos de procesamiento por lotes que posibilitaron la ejecuci�n de decenas de experimentos en una sola orden y 
\item La visi�n de que en alg�n momento del desarrollo futuro el sistema pueda ser embebido a un microcontrolador o alg�n otro hardware especializado.
\end{enumerate} 

\section{M�dulo Optik}\label{sec:optik}
Optik es un software que genera bases de datos de im�genes de objetos
destinadas para entrenamiento y pruebas del clasificador neuronal
LIRA. Se parte de im�genes generales de m�ltiples objetos ordenados
aleatoriamente, con ayuda de marcas colocadas por el usuario, genera
una base de datos de im�genes normalizadas mediante un proceso de
extracci�n. im�genes normalizadas se refiere a que �stas tienen
iguales caracter�sticas: contienen una pieza centrada y con
orientaci�n fija de 0� respecto a su eje mayor, son de dimensi�n
constante y tienen una ventana circular que facilita la rotaci�n
(Fig. \ref{optikpantallaconmuestrasetiquetasymarcas}). Las im�genes extra�das son nombradas de acuerdo al tipo de
pieza correspondiente y a un n�mero consecutivo. Tambi�n el sistema
crea un archivo MRK que contiene los datos de todas las marcas
realizadas.

\begin{figure}[h]
\begin{center} \includegraphics[width=10cm]
{figuras/optikPantallaConMuestrasEtiquetasYMarcas.jpg}\end{center}
\caption[Interfaz gr�fica de usuario (IGU) de Optik.]{IGU Optik, la cual permite marcar y extraer las muestras de im�genes a partir de las im�genes escena.}
\label{optikpantallaconmuestrasetiquetasymarcas}
\end{figure}

Antes de el proceso de colocaci�n de muestras, deben definirse los
nombres y otras propiedades de los distintos tipos de piezas con que
se va a trabajar (i. e. dimensiones, peso y material), �stos datos son
de utilidad al sistema para calcular autom�ticamente el centro de la
pieza adem�s de futuras operaciones. Esta informaci�n se almacena en
el sistema en un archivo PZA y puede ser cargada cuando se
requiera. Optik adem�s realiza algunas tareas de preprocesamiento de
im�genes, como extracci�n de contornos y conversi�n de im�genes de
color a escala de grises.

Debido al hecho de que las marcas en las escenas son puestas por el usuario mediante el rat�n, las im�genes muestras para las bases de datos no est�n centradas ni orientadas exactamente. Este no es un problema ya que los clasificadores utilizados son capaces de llevar a cabo la tarea de reconocimiento con muestras imperfectamente centradas o giradas, sobre todo en conjuntos grandes de entrenamiento.

Este software en primera instancia se elabor� con Borland$^{MR}$ C++, luego se decidi� en colaboraci�n con otros colegas pasarlo a sistema operativo GNU/Linux \cite{Josue2006}.

\section{M�dulo RNA}
El m�dulo RNA implementa al clasificador neuronal LIRA, es decir, es
la realizaci�n de la red neuronal completa, neurona a neurona, sus
interconexiones y toda su funcionalidad. Este es el m�dulo m�s
elaborado, su eficiencia es cr�tica, por lo que fue necesario cuidar
dos aspectos fundamentales, memoria y velocidad. Un clasificador
neuronal puede necesitar cientos de miles de neuronas y una magnitud
mayor de interconexiones adem�s de ser indispensable tiempos de
ejecuci�n total aceptables, tanto en las fases de entrenamiento como
en las de prueba, siendo cr�tico en una aplicaci�n real de
reconocimiento (m�dulo Localizador). Para cuidar la velocidad se
decidi� programar este m�dulo y por extensi�n todo OptikRNA con
lenguaje C++, es decir, se utiliz� la velocidad y el poder de C
combinado con las ventajas de la POO. Se utilizaron punteros, estos
permiten operar a bajo nivel mejorando la velocidad y realizar las
interconexiones entre las distintas clases utilizadas m�s
eficientemente. Para cuidar la eficiencia evitando operaciones de
punto flotante se utilizaron �nicamente n�meros enteros. Este m�dulo
se constituye de diversas clases, las principales, junto con su
relaci�n de herencia se muestran en la Fig. \ref{RNAclasesYHerencia}.

\begin{figure}[h]
\begin{center}
\includegraphics[
%natheight=1.760800in,
%natwidth=2.603900in,
%height=1.2756in,
width=4in]
{figuras/RNAclasesYHerencia.png}
\caption[Clases principales del m�dulo RNA y su relaci�n de herencia.]{Clases principales del m�dulo RNA y su relaci�n de herencia. a) Clases a nivel neurona. b) Clases a nivel capa.}\label{RNAclasesYHerencia}
\end{center}
\end{figure}

Existe una s�per clase llamada RNA la cual construye con las clases
descritas antes y otras menores no mencionadas el clasificador
neuronal LIRA. Esta clase ofrece las mismas funciones que tiene el
clasificador Lira descritas en la Secci�n 2 adem�s de otras necesarias
para su implementaci�n y funcionamiento, entre las principales est�n:
creaci�n, codificaci�n, entrenamiento, reconocimiento y borrado de
memoria (pesos a cero). La clase RNA controla completamente a todo el
m�dulo y es con la cu�l se comunica realmente el m�dulo Interfaz. 

\section{M�dulo Localizador}
El m�dulo Localizador se encarga de buscar una pieza requerida en una
imagen fuente y definir la posici�n de �sta. La pieza que ser� capaz
de localizar este m�dulo debe estar en el conjunto de piezas con que
el clasificador se haya entrenado previamente. La imagen fuente no
tiene por que ser una imagen ocupada para la extracci�n de im�genes
normalizadas, puede ser cualquier imagen siempre que contenga el
objeto a buscar en la misma escala en que existe en las im�genes
ocupadas.

Sobre una imagen dada, este m�dulo aplica un algoritmo de b�squeda que
consiste en ir tomando subim�genes empezando por el centro y
desplaz�ndose en forma espiral (Fig. 5b). Para cada posici�n se pide
al m�dulo RNA identificar la pieza requerida y si no se encuentra se
rota un cierto �ngulo y se repite el proceso hasta encontrar la pieza
o cuando una rotaci�n es completada, si no se encuentra, se continua
el desplazamiento espiral hasta encontrar la pieza buscada o alcanzar
los l�mites de la imagen. Los desplazamientos lineales y angulares
pueden ser definidos por el usuario.

El m�dulo Localizador constituye una aplicaci�n pr�ctica concreta en
microensamble y puede funcionar independientemente de los m�dulos
Optik e Interfaz con el fin de ser acoplado a sistemas autom�ticos de
manipulaci�n de piezas. En la Fig. 5c se muestra un resultado de la
b�squeda de una pieza ("tornillo de cabeza redonda") sobre una imagen
que contiene diversas piezas. El sistema despliega las coordenadas de
la pieza as� como la orientaci�n. Cuando la pieza se localiza lejos de
su centro, la orientaci�n es err�nea, por esta raz�n este m�dulo debe ser mejorado.

\section{Interfaz}
El m�dulo Interfaz es el �nico m�dulo que se comunica con el usuario,
adem�s intercomunica los dem�s m�dulos para administrarlos. Este
m�dulo en si es una Interfaz Gr�fica de Usuario (IGU), cuenta con
todas las funciones disponibles de OptikRNA y con �reas para desplegar
im�genes y resultados. En la Fig. \ref{redneuartpantallapruebaeinfo}
se muestra �sta interfaz. El �rea llamada "par�metros del clasificador" es a donde el
usuario ingresa los par�metros para la creaci�n de un clasificador
neuronal LIRA en particular. Abajo est� el panel de funciones b�sicas
de la RNA, desde aqu� se env�an los comandos para el m�dulo RNA, como
crear, guardar, cargar y reconocer. En el �rea de mensajes se
despliega informaci�n general del clasificador cargado. En el panel de
funciones avanzadas existen funciones para el m�dulo RNA como
entrenar, probar, asignar bases de datos y entrenar-probar
autom�ticamente. En el �rea de resultados se muestran los resultados
de las pruebas o el reconocimiento de piezas. Por �ltimo, desde el
panel del m�dulo Localizador accesamos a las funciones y par�metros de
este m�dulo. En la figura referida se muestra un clasificador cargado
junto con su informaci�n general y los resultados de una prueba
aplicada. En el �rea vac�a es donde se despliega la imagen utilizada
por el Localizador. Las funciones del m�dulo Optik est�n en otra IGU que no se muestra.

\begin{figure}[h]
\begin{center} \includegraphics[width=15cm]{figuras/redneuartPantallaPruebaEInfoV2.jpg}\end{center}
\caption{Interfaz gr�fica de usuario del software RNA.}
\label{redneuartpantallapruebaeinfo}
\end{figure}


\begin{figure}[h]
\begin{center}
\includegraphics[natheight=1.760800in,natwidth=2.603900in,height=1.2756in,width=1.8784in]{figuras/blockDiagramRedneuartOptikV2.jpg}
\caption{Diagrama a bloques del software OptikRNA.}\label{blockdiagramredneuartoptik}
\end{center}
\end{figure}

El usuario trabaja con el m�dulo de software de redes neuronales que implementa el clasificador LIRA, esto lo hace a trav�s de la IGU Rna (Fig. \ref{redneuartpantallapruebaeinfo}). Con la ayuda de esta interfaz el usuario puede crear el clasificador neuronal LIRA, entrenar, probar y usar el clasificador as� como utilizar el localizador de piezas. La IGU Rna con un clasificador entrenado ya cargado es mostrada en la Fig. \ref{redneuartpantallapruebaeinfo}. En el cuadro de texto de la izquierda se muestra informaci�n sobre el clasificador. Los resultados se dan en el cuadro de texto de la derecha: el porcentaje de reconocimiento obtenido y los nombres de los archivos de las muestras que el sistema no pudo reconocer.

En el centro de la Fig. \ref{redneuartpantallabusquedaehistograma} se muestra una escena en la cu�l se ha llevado a cabo el procesamiento con el localizador de piezas. Un clasificador previamente entrenado y probado ha sido ya cargado. En esta misma imagen se muestra tambi�n en el cuadro de texto de la derecha, una marca sobre la pieza localizada as� como las coordenadas y orientaci�n asociada a la misma.

\begin{figure}[h]
\begin{center} \includegraphics[width=15cm
]{figuras/redneuartPantallaBusquedaEHistograma.jpg}\end{center}
\caption{IGU Rna. Localizador de piezas.}
\label{redneuartpantallabusquedaehistograma}
\end{figure}

\section{Implementaci�n de LIRA}
Our neural classifier software is based on integer numbers operations in order
to avoid large time effort that is necessary for floating point operations.
The most general classes were made, that means that the same software modules
can be used to create different neural network topologies. The user
interface was made specially for LIRA neural classifier. It
is not hard to use this software modules to construct other types
of neural networks or make changes to the current topology, e. g.
add more layers. 

The most important classes created for artificial neural network realization
were: Dentrita, Neuron, NeuralSet and Rna.

The Dentrita class is a basic one. It has only two attributes, stimulus
and weight. Dentrita instances are used widely by neuron objects in
order to create fully functional neurons. In Fig. \ref{classneuron}
the Neuron class and its derived classes are shown. The neuron types used in LIRA classifier
are ON, OFF, AND and adding neurons. The ON and OFF neurons are one
input (OI) neurons and the rest are several input (SI) neurons. The
Neuron class is the fundamental part of the neural network software.

\begin{figure}[h]
\begin{center} \includegraphics[
natheight=2.000300in,
natwidth=4.125200in,
height=2.0384in,
width=4.1753in
]{figuras/classNeuron.jpg}\end{center}
\caption{Clase neurona y sus clases derivadas.}
\label{classneuron}
\end{figure}

In order to construct neuron sets a general NeuralSet class was
created. Rna is the class that combines all mentioned classes. This class we use to
construct a LIRA neural network. Examples of the parameters
are the number of neurons or groups in each layer, the number of ON
and OFF neurons in each group, the input vector and the output classes.
The Rna object contains information about itself. It is possible to store
the complete neural network including all its internal parameters,
to load a previously stored neural network and to perform training
and recognition processes. 

The neural classifier is controlled by an object called
Rna-Interface. That interrelates the user by means the GUI with the databases used for training and testing of the classifier. 

\section{Software de l�nea de comandos}
\subsection{lira2007}
A continuaci�n se presenta la ayuda propia del software \emph{lira2007} que ha sido usado para implementar y experimentar con el clasificador LIRA as� como su interacci�n con las bases de datos.

\begin{verbatim}
Example to use follow()-functions:

USAGE:
--help, -h
       display this help.

*** Setup ***
--create adn='<RNA name (string)> <tamVectEnt> <ImageWidth> <windowWidth> <windo
wHeight> <numGroup> <elemXGroup> <numNeuON> <numNeuOFF> <eta (double)> <numClass
es>', you can use -c instead of --create. Non especified are (int).
For example use: 
	 lira -c adn='liraSUN 22500 150 15 15 170000 7 4 3 1.0 8'  
       Create a Lira Neural Classifier from scrath.
--load <filename.rna>, -l <filename.rna>
       load a neural classifier from specified file.
--info, -i
       displays info about the loaded classifier.
--classes-file <filename.ent>, -cf <filename.ent>
       assign the classes file to LIRA classifier.

*** Preparing ***
--train-dir <dir>, -td <dir>
       assign the training directory, default is "./train".
--code-dir <dir>, -cd <dir>
       assign the code directory, default is "./code".
--code, -k
       code all the images in the train directory, they will put in the code di
rectory
--reset, -kill
       reset the internal connections of the loaded ANN, Erase its memory
--train num, -t num
       train using the coded images from the given code dir using "num" cycles.
 Default is 40
--prove-dir <dir>, -pd <dir>
       assign the prove directory, default is "./prove".
--images-dir <dir>, -id <dir>
       assign the images directory, default is "./images".
--train-percentage num, -tp num
       Percentaje of images in images-dir to be selected for the training set. 
Default is 50%

*** Using ***
--prove, -p
       proves the classifier with the images files stored in the "proving direc
tory".
--recognize <filename.png>, -r <filename.png>
       recognize a class in the given normalized image file.
pos='<x (int)> <y (int)> <O (double)>', Parameters especification for recogniz
ed a big (not normalized) image in certain point and orientation. All values s
hould be (int).
You might specified a "filename.png" by -r to look for.
Example use: 
	 lira -l lira.rna -r imagen.png pos='10 20 -45.0' 
An image file called "cut.png" will be created with the used subimage.

*** Ending***
--save, -s
       Save the current classifier to a .rna file.
--close, -q
       (NOW FAIL!) Save the current classifier to a .rna file.
       close the current classifier loaded in memory.
--output, -o
       Print values from the output layer.

*** Search ***
--searchImage <filename.png>, -si <filename.png>
       Search for some recognized clase.
--searchClass <className>, -sc <className>
       Class to be search by the Search command. Use "prueba" to make a prove
 and "cualquiera" to search anyone. Default is anyone
--searchQuantity num, -sq num
       Number of objects to search for. Default is 1.
--searchStep num, -ss num
       Step in pixels for the searcher to jump (/\x & /\y). Default is 20.
--searchStepAngle num, -sa num
       Angle in degrees for the searcher to jump. Default is 45.

*** Tools ***
--distortion dis='<dx> <nx> <dy> <ny> <dO> <nO>', you can use -d instead of -
-distortion. All parameters should be (int).
You might specified the "training dir".
For example use: 
	 lira -td ./trainimages -d dis='5 2 5 2 5 2' 
\end{verbatim}

\subsection{pcnc2007}
A continuaci�n se presenta la ayuda propia del software \emph{pcnc2007} que ha sido usado para implementar y experimentar con el PCNC as� como su interacci�n con las bases de datos.

\begin{verbatim}
Example to use follow()-functions:

USAGE:
--help, -h
       display this help.

*** Setup ***
--create adn='<PCNC name (string)> <method> <Tmin> <Tmax> <w> <h> <p> <n> <S>
 <N> <K> <Dc> <q> <numClasses>', you can use -c instead of --create. Non espe
cified are (int).
For example use: 
	 peco -c adn='PCNCprueba 0 1 32000 5 5 3 2 500 1000 12 8 5 8'  
       Create a Permutative Code Neural Classifier (PCNC) from scrath.
--load <filename.pcnc>, -l <filename.pcnc>
       load a PCNC from specified file.
--info, -i
       displays info about the loaded PCNC.
--classes-file <filename.ent>, -cf <filename.ent>
       assign the classes file to PCNC.

*** Preparing ***
--train-dir <dir>, -td <dir>
       assign the training directory, default is "./train".
--code-dir <dir>, -cd <dir>
       assign the code directory, default is "./code".
--code, -k
       code all the images in the train directory, they will put in the code 
directory
--reset
       reset the internal connections of the loaded PCNC, Erase its memory
--train num, -t num
       train using the coded images from the given code dir using "num" cycles
. Default is 40
--prove-dir <dir>, -pd <dir>
       assign the prove directory, default is "./prove".
--images-dir <dir>, -id <dir>
       assign the images directory, default is "./images".
--train-percentage num, -tp num
       Percentaje of images in images-dir to be selected for the training set.
 Default is 50%

*** Using ***
--prove, -p
       proves the PCNC with the images files stored in the "proving directory".
--recognize <filename.png>, -r <filename.png>
       recognize a class in the given normalized image file.

*** Ending***
--save, -s
       Save the current PCNC to a .pcnc file.
--close, -q
       (NOW FAIL!) Save the current PCNC to a .pcnc file.
       close the current PCNC loaded in memory.
\end{verbatim}

\subsection{Potencial para la experimentaci�n}
Como un ejemplo del uso ventajoso del software de l�nea de comandos se presenta uno de los m�ltiples archivos de procesamiento por lotes\footnote{M�s conocidos por su denominaci�n en ingl�s: \emph{scripts}} empleado para realizar decenas de experimentos con cambi� de par�metros de forma autom�tica, es decir, en una sola corrida.

\begin{verbatim}
#!/bin/bash

#Decenas de pruebas para estudiar comportamiento de par�metros
#La base es el Mejor:
#../../bin/pcnc2007 -c adn='PCNCmejorBD-A 1 0 65535 10 10 5 4 1000 300000 20 5
 2 8' -cf BD-A.ent -td ../../muestras/entrenamiento -cd ../../muestras/codigo 
-pd ../../muestras/prueba -k -t 40 -p -s


#*************PARAMETRO W
#Inicializacion
PARAM=W
PARw=10
PARp=5
PARn=4
PARS=1000
PARN=300000
PARK=20
PARDc=5
PARq=2
#Usando "for" para probar distintos par�metros 
for PARw in 05 08 09 11 12 15 20;
  do
  echo "**************************"
  echo "***** CORIDA: "$PARAM:$PARw" *******"
  echo "**************************"
  #Hace todo de una vez, tambi�n salva
  ../../bin/pcnc2007 -c adn='PCNCparam'$PARAM''$PARw' 1 0 65535 '$PARw' '$PARw
' '$PARp' '$PARn' '$PARS' '$PARN' '$PARK' '$PARDc' '$PARq' 7' -cf BD-D.ent -td
 ../../muestrasBD-D/entrenamiento -cd ../../muestrasBD-D/codigo -pd ../../mues
trasBD-D/prueba -k -t 30 -p -s

# ../../bin/pcnc2007 -c adn='PCNCparam'$PARAM' 1 0 65535 '$PARw' '$PARw' '$PAR
p' '$PARn' '$PARS' '$PARN' '$PARK' '$PARDc' '$PARq' 7' -cf BD-D.ent -i
done  

#*************PARAMETRO K
#Inicializacion
PARAM=K
PARw=10
PARp=5
PARn=4
PARS=1000
PARN=300000
PARK=20
PARDc=5
PARq=2
#Usando "for" para probar distintos par�metros 
for PARK in 05 10 15 25 30;
  do
  echo "**************************"
  echo "***** CORIDA: "$PARAM:$PARK" *******"
  echo "**************************"
  #Hace todo de una vez, tambi�n salva
  ../../bin/pcnc2007 -c adn='PCNCparam'$PARAM''$PARK' 1 0 65535 '$PARw' '$PARw
' '$PARp' '$PARn' '$PARS' '$PARN' '$PARK' '$PARDc' '$PARq' 7' -cf BD-D.ent -td
 ../../muestrasBD-D/entrenamiento -cd ../../muestrasBD-D/codigo -pd ../../mues
trasBD-D/prueba -k -t 30 -p -s
done  

#*************PARAMETRO Dc
#Inicializacion
PARAM=Dc
PARw=10
PARp=5
PARn=4
PARS=1000
PARN=300000
PARK=20
PARDc=5
PARq=2
#Usando "for" para probar distintos par�metros 
for PARDc in 02 04 06 08 10 15 20 25;
  do
  echo "**************************"
  echo "***** CORIDA: "$PARAM:$PARDc" *******"
  echo "**************************"
  #Hace todo de una vez, tambi�n salva
  ../../bin/pcnc2007 -c adn='PCNCparam'$PARAM''$PARDc' 1 0 65535 '$PARw' '$PARw
' '$PARp' '$PARn' '$PARS' '$PARN' '$PARK' '$PARDc' '$PARq' 7' -cf BD-D.ent -td
 ../../muestrasBD-D/entrenamiento -cd ../../muestrasBD-D/codigo -pd ../../mues
trasBD-D/prueba -k -t 30 -p -s
done  

#*************PARAMETRO q
#Inicializacion
PARAM=q
PARw=10
PARp=5
PARn=4
PARS=1000
PARN=300000
PARK=20
PARDc=5
PARq=2
#Usando "for" para probar distintos par�metros 
for PARq in 00 01 03 04 05 10;
  do
  echo "**************************"
  echo "***** CORIDA: "$PARAM:$PARq" *******"
  echo "**************************"
  #Hace todo de una vez, tambi�n salva
  ../../bin/pcnc2007 -c adn='PCNCparam'$PARAM''$PARq' 1 0 65535 '$PARw' '$PARw
' '$PARp' '$PARn' '$PARS' '$PARN' '$PARK' '$PARDc' '$PARq' 7' -cf BD-D.ent -td
 ../../muestrasBD-D/entrenamiento -cd ../../muestrasBD-D/codigo -pd ../../mues
trasBD-D/prueba -k -t 30 -p -s
done  

#*************PARAMETROS p y q
#Inicializacion
PARAM=PyN
PARw=10
PARp=5
PARn=4
PARS=1000
PARN=300000
PARK=20
PARDc=5
PARq=2
#Usando "for" para probar distintos par�metros 
for PARpn in "3 6" "4 5" "6 3" "7 2" "2 5" "3 4" "4 3" "5 2" "6 1" "4 7" "5 6"
 "6 5" "7 4" "8 3" "9 2";
  do
  echo "**************************"
  echo "***** CORIDA: "$PARAM:$PARpn" *******"
  echo "**************************"
  #Variable auxiliar para nombrar sin espacios
  PARpnNom=${PARpn/ /y}
  #Hace todo de una vez, tambi�n salva
  ../../bin/pcnc2007 -c adn='PCNCparam'$PARAM''$PARpnNom' 1 0 65535 '$PARw' '$P
ARw' '"$PARpn"' '$PARS' '$PARN' '$PARK' '$PARDc' '$PARq' 7' -cf BD-D.ent -i -td
 ../../muestrasBD-D/entrenamiento -cd ../../muestrasBD-D/codigo -pd ../../mues
trasBD-D/prueba -k -t 30 -p -s
done  



#*************PARAMETRO N
#Inicializacion
PARAM=N
PARw=10
PARp=5
PARn=4
PARS=1000
PARN=300000
PARK=20
PARDc=5
PARq=2
#Usando "for" para probar distintos par�metros 
for PARN in 100000 200000 250000 400000 500000;
  do
  echo "**************************"
  echo "***** CORIDA: "$PARAM:$PARN" *******"
  echo "**************************"
  #Hace todo de una vez, tambi�n salva
  ../../bin/pcnc2007 -c adn='PCNCparam'$PARAM''$PARN' 1 0 65535 '$PARw' '$PARw
' '$PARp' '$PARn' '$PARS' '$PARN' '$PARK' '$PARDc' '$PARq' 7' -cf BD-D.ent -td
 ../../muestrasBD-D/entrenamiento -cd ../../muestrasBD-D/codigo -pd ../../mues
trasBD-D/prueba -k -t 30 -p -s
done  


#*************PARAMETRO S
#Inicializacion
PARAM=S
PARw=10
PARp=5
PARn=4
PARS=1000
PARN=300000
PARK=20
PARDc=5
PARq=2
#Usando "for" para probar distintos par�metros 
for PARS in 0400 0600 0800 1200 1500 2000 2500 3000 4000 5000 6000 10000;
  do
  echo "**************************"
  echo "***** CORIDA: "$PARAM:$PARS" *******"
  echo "**************************"
  #Hace todo de una vez, tambi�n salva
  ../../bin/pcnc2007 -c adn='PCNCparam'$PARAM''$PARS' 1 0 65535 '$PARw' '$PARw
' '$PARp' '$PARn' '$PARS' '$PARN' '$PARK' '$PARDc' '$PARq' 7' -cf BD-D.ent -td
 ../../muestrasBD-D/entrenamiento -cd ../../muestrasBD-D/codigo -pd ../../mues
trasBD-D/prueba -k -t 30 -p -s
done  

#Crea el resumen de resultados
grep "El porcentaje" PCNCparam*.proveresults >> resumen.proveresults

#FIN

#  LocalWords:  PARw
 
\end{verbatim}

%%% Local Variables: 
%%% mode: latex
%%% TeX-master: "tesis"
%%% End: 

\chapter{Publicaciones}
El presente trabajo di� lugar a las siguientes publicaciones y congresos nacionales e internacionales:

\begin{enumerate}
\item G.K. Toledo, E. Kussul, T. Baidyk. Neural classifier LIRA for recognition of micro work pieces and their positions in the processes of microassembly and micromanufacturing. The Seventh All-Ukrainian International Conference on Signal/Image Processing and Pattern Recognition, UkrObraz2004, Kiev, Ukraine, October 2004, pp. 17�20.
\item G. Toledo, E. Kussul, T. Baidyk. Clasificador neuronal LIRA para reconocimiento de piezas de trabajo y sus posiciones en los procesos de microensable y micromanufactura. Memorias del XIX Congreso de Instrumentaci�n SOMI, Pachuca, Hidalgo, M�xico, 25-29, octubre, 2004.
\item Ernst Kussul, Tatiana Baidyk, Gengis Kanhg Toledo Ram�rez. Investigaci�n y desarrollo del sistema de visi�n t�cnica para aplicaciones en micromaquinado y microensamble. Reporte t�cnico, CCADET-UNAM, 15 de junio de 2005, 29 p.
\item Gengis K. Toledo-Ram�rez. Software orientado a objetos para investigaci�n de redes neuronales. Memorias del XX Congreso de Instrumentaci�n SOMI, Le�n, Guanajuato, M�xico, 24-28, Octubre, 2005.
%2006
\item Gengis K. Toledo, Ernst Kussul, Tatiana Baidyk. Neural classifier for micro work piece recognition. Image and Vision Computing. Elsevier. Vol 24/8, 2006, pp 827-836.
\item Gengis K. Toledo-Ram�rez, Ernst Kussul, Tatiana Baidyk. Object oriented software for micro work piece recognition in microassembly. Journal of Applied Research and Technology. Vol 4/1, 2006, pp 59-74.
\item Josu� Enr�quez Z�rate, Zaizar Betanzos Cort�s, Gengis Kanhg Toledo Ram�rez. Sistema Generador de Bases de Datos para Im�genes: una aplicaci�n con Gambas. I Simposium de Linux de la Mixteca, 2-4, Marzo, 2006. http://www.utm.mx/gulmix/simposium\_2006/ponencia.html
\item Gengis K. Toledo, Ernst Kussul, Tatiana Baidyk. Clasificaci�n de piezas mediante un clasificador neuronal artificial. Memorias del XXI Congreso de Instrumentaci�n SOMI, Ensenada, Baja California, M�xico, 22-25, Octubre, 2006.
%2007
\item Gengis Kanhg, Kussul Ernst, Baidyk Tatiana. Reconocimiento de piezas mediante un clasificador neuronal con permutaci�n de c�digos. Art�culo aceptado para el XXII Congreso de Instrumentaci�n SOMI, Monterrey, Nuevo Le�n, M�xico, 1-4, Octubre, 2007.
\end{enumerate}

Asi mismo al momento de terminar este trabajo una nueva publicaci�n internacional est� en proceso y un nuevo trabajo est� postul�ndose para un congreso internacional.

%%% Local Variables: 
%%% mode: latex
%%% TeX-master: "tesis"
%%% End: 



%%%%%%%%%%%%%
\backmatter
%%%%%%%%%%%%%
% Adjustments headers
\fancyhead[RO]{\leftmark}
\fancyhead[EL]{}
\addcontentsline{toc}{chapter}{Bibliograf�a}
\bibliographystyle{unsrt}
\bibliography{lmm,gengisNOlmm}
%\printindex
\end{document}

%%% Local Variables: 
%%% mode: latex
%%% TeX-master: "tesis"
%%% End: 
