%%% Documento realizada con LATEX, GNU, etc.
\newpage
\begin{flushright}

$ $

\vspace{18.6cm}

$ $

\rule[0mm]{4.6cm}{0.2mm}

Redacci�n y edici�n de tesis 

con \LaTeXe, \emph{GNU-Emacs} 

y sistema operativo libre 

%\emph{GNU/Linux}.

\emph{
\raisebox{-0.6ex}{G}
\raisebox{-0.1ex}{N}
\raisebox{0.3ex}{U}
\raisebox{0.0ex}{/}
\raisebox{-0.3ex}{L}
\raisebox{-0.6ex}{I}
\raisebox{-0.2ex}{N}
\raisebox{0.3ex}{U}
\raisebox{0.6ex}{X}
.}

\end{flushright}

%%%%%%% Dedicatoria
\clearfullypage
\vspace*{8cm} 

\begin{center}

\hspace{1ex} A Susy y Conenetl;

\hspace{1ex} \small{mi familia} 

\hspace{1ex} $\heartsuit$

\end{center}

%%%%%%%%% Cita
\clearfullypage
\vspace*{5cm} 

%\hspace{1ex} 
\begin{quotation} Vivimos en una sociedad altamente dependiente de la ciencia y de la tecnolog�a, en la cual casi nadie sabe nada sobre ciencia y tecnolog�a.
\flushright \emph{Carl Sagan}
\end{quotation}

\vspace*{1cm} 

\begin{quotation} Se ha vuelto espantosamente obvio que nuestra tecnolog�a ha excedido a nuestra humanidad.
\flushright \emph{Albert Einstein}
\end{quotation}

\vspace*{1cm} 
 
\begin{quotation} Las m�quinas y las computadoras deber�n volverse una parte funcional en un sistema social orientado por la vida y no un c�ncer que empieza por hacer estragos y acaba por matar al sistema.
\flushright \emph{Erich Fromm}
\end{quotation}






%%% Local Variables: 
%%% mode: latex
%%% TeX-master: "tesis"
%%% End: 
